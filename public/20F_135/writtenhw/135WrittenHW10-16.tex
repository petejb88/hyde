\documentclass[11pt,reqno,final]{amsart}

\pdfcompresslevel=0
\pdfobjcompresslevel=0

\usepackage[dvipsnames]{xcolor}% adds colors
\usepackage{amsmath, amsthm}% {amsfonts, amssymb}

% New Characters
\usepackage[latin1]{inputenc}%
\usepackage[T1]{fontenc}

\usepackage{MnSymbol}
\usepackage[normalem]{ulem}% underlining

\usepackage[theoremfont, largesc]{newpxtext} % different text,math font
\usepackage{newpxmath}

\makeatletter
\DeclareMathRadical{\sqrtsign}{symbols}{112}{largesymbols}{112}
\let\sqrt=\undefined
\DeclareRobustCommand\sqrt{\@ifnextchar[\@sqrt{\mathpalette\@x@sqrt}}
\def\@x@sqrt#1#2{%
 \setbox\z@\hbox{$\m@th#1\sqrtsign{\mkern1mu #2}$}
 \mkern3mu\box\z@}
\makeatother




% Page Typesetting
\usepackage[final]{microtype}
\usepackage{relsize}
\usepackage[margin=1in]{geometry}
\usepackage{framed}
\usepackage{tikz}
\usepackage{setspace}
\onehalfspacing

\usepackage{hyperref}
\hypersetup{
  final,
  pdftitle={Math 135 - Written HW 10-17},
  pdfauthor={Bonventre}, 
  linktoc=page,
  pagebackref,
  colorlinks=true,
  citecolor=PineGreen,
  linkcolor=PineGreen,
  linkbordercolor=PineGreen,
}


% Internal References

\usepackage[inline,shortlabels]{enumitem}

\numberwithin{equation}{section} 
\numberwithin{figure}{section}

\usepackage[nameinlink,capitalise,noabbrev]{cleveref}

\crefname{equation}{}{} % get \cref to behave as \eqref

% \theoremstyle{plain} % bold name, italic text
\newtheorem{theorem}[equation]{Theorem}%
\newtheorem*{theorem*}{Theorem}%
\newtheorem{lemma}[equation]{Lemma}%
\newtheorem{proposition}[equation]{Proposition}%
\newtheorem{corollary}[equation]{Corollary}%
\newtheorem{conjecture}[equation]{Conjecture}%
\newtheorem*{conjecture*}{Conjecture}%
\newtheorem{claim}[equation]{Claim}%
\newtheorem{question}{Question}

\theoremstyle{definition} % bold name, plain text
\newtheorem{definition}[equation]{Definition}%
\newtheorem*{definition*}{Definition}%
\newtheorem{example}[equation]{Example}%
\newtheorem*{example*}{Example}%
\newtheorem{remark}[equation]{Remark}%
\newtheorem{notation}[equation]{Notation}%
\newtheorem{convention}[equation]{Convention}%
\newtheorem{assumption}[equation]{Assumption}%
\newtheorem{exercise}[question]{Exercise}

% ---------- macros
\newcommand{\set}[1]{\left\{#1\right\}}%
\newcommand{\sets}[2]{\left\{ #1 \;|\; #2\right\}}%
\newcommand{\longto}{\longrightarrow}%
\newcommand{\into}{\hookrightarrow}%
\newcommand{\onto}{\twoheadrightarrow}%

\usepackage{harpoon}
\newcommand{\vect}[1]{\text{\overrightharp{\ensuremath{#1}}}}

\newcommand{\del}{\partial}%

\newcommand{\ki}{\chi}
\newcommand{\ksi}{\xi}
\newcommand{\Ksi}{\Xi}

\newcommand{\dlim}{\displaystyle\lim}

% %%%%%%%%%%%%%%%%%%%%%%%%%%%%%%%%%%%%%%%%%%%%%%%%%%%%%%%%%%%%%%%%%%%%%%%%%%%%%%%%%%%%%%%%%%%%%%%%%%%%

\begin{document}

\begin{center}
        \textbf{\Large Math 135, Calculus 1, Fall 2020}\\[10pt]
        {\large Written Homework 10-17}
\end{center}

\thispagestyle{empty}

\renewcommand{\thesection}{\Alph{section}}

\subsection*{Directions:}
Write your solutions neatly and clearly, and submit to Canvas.
In these problems, you should show all of your work in complete mathematical "sentences", writing complete English sentences when you explain your logic.
You are free (and encouraged!) to work with others, but make sure the solutions you write up your solutions indepedently.

\begin{exercise}
        Evaluate the limit $\dlim_{x \to 0}\dfrac{\sin(8x)}{\sin(9x)}$.
\end{exercise}

\begin{exercise}
        Sketch the graph of a single differentiable function $f$ for which:
        \begin{itemize}
        \item $f(0) = 0$
        \item $f'(0) = 3$
        \item $f'(1) = 0$
        \item $f'(2) = -1$
        \item $\dlim_{x \to \infty} = 0$
        \item $\dlim_{x \to -\infty} = -2$
        \end{itemize}
\end{exercise}

\begin{exercise}
        Find the equations for the two distinct tangent lines to the curve $y = x^2$ that pass through the point $(-1,-3)$.
\end{exercise}

\end{document}


