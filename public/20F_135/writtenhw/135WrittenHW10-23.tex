\documentclass[11pt,reqno,final]{amsart}

\pdfcompresslevel=0
\pdfobjcompresslevel=0

\usepackage[dvipsnames]{xcolor}% adds colors
\usepackage{amsmath, amsthm}% {amsfonts, amssymb}

% New Characters
\usepackage[latin1]{inputenc}%
\usepackage[T1]{fontenc}

\usepackage{MnSymbol}
\usepackage[normalem]{ulem}% underlining

\usepackage[theoremfont, largesc]{newpxtext} % different text,math font
\usepackage{newpxmath}

\makeatletter
\DeclareMathRadical{\sqrtsign}{symbols}{112}{largesymbols}{112}
\let\sqrt=\undefined
\DeclareRobustCommand\sqrt{\@ifnextchar[\@sqrt{\mathpalette\@x@sqrt}}
\def\@x@sqrt#1#2{%
 \setbox\z@\hbox{$\m@th#1\sqrtsign{\mkern1mu #2}$}
 \mkern3mu\box\z@}
\makeatother




% Page Typesetting
\usepackage[final]{microtype}
\usepackage{relsize}
\usepackage[margin=1in]{geometry}
\usepackage{framed}
\usepackage{tikz}
\usepackage{setspace}
\onehalfspacing

\usepackage{hyperref}
\hypersetup{
  final,
  pdftitle={Math 135 - Written HW 10-23},
  pdfauthor={Bonventre}, 
  linktoc=page,
  pagebackref,
  colorlinks=true,
  citecolor=PineGreen,
  linkcolor=PineGreen,
  linkbordercolor=PineGreen,
}


% Internal References

\usepackage[inline,shortlabels]{enumitem}

% \numberwithin{equation}{section} 
\numberwithin{figure}{section}

\usepackage[nameinlink,capitalise,noabbrev]{cleveref}

\crefname{equation}{}{} % get \cref to behave as \eqref

% \theoremstyle{plain} % bold name, italic text
\newtheorem{theorem}[equation]{Theorem}%
\newtheorem*{theorem*}{Theorem}%
\newtheorem{lemma}[equation]{Lemma}%
\newtheorem{proposition}[equation]{Proposition}%
\newtheorem{corollary}[equation]{Corollary}%
\newtheorem{conjecture}[equation]{Conjecture}%
\newtheorem*{conjecture*}{Conjecture}%
\newtheorem{claim}[equation]{Claim}%
\newtheorem{question}{Question}

\theoremstyle{definition} % bold name, plain text
\newtheorem{definition}[equation]{Definition}%
\newtheorem*{definition*}{Definition}%
\newtheorem{example}[equation]{Example}%
\newtheorem*{example*}{Example}%
\newtheorem{remark}[equation]{Remark}%
\newtheorem{notation}[equation]{Notation}%
\newtheorem{convention}[equation]{Convention}%
\newtheorem{assumption}[equation]{Assumption}%
\newtheorem{exercise}[question]{Exercise}

% ---------- macros
\newcommand{\set}[1]{\left\{#1\right\}}%
\newcommand{\sets}[2]{\left\{ #1 \;|\; #2\right\}}%
\newcommand{\longto}{\longrightarrow}%
\newcommand{\into}{\hookrightarrow}%
\newcommand{\onto}{\twoheadrightarrow}%

\usepackage{harpoon}
\newcommand{\vect}[1]{\text{\overrightharp{\ensuremath{#1}}}}

\newcommand{\del}{\partial}%

\newcommand{\ki}{\chi}
\newcommand{\ksi}{\xi}
\newcommand{\Ksi}{\Xi}

\newcommand{\dlim}{\displaystyle\lim}

% %%%%%%%%%%%%%%%%%%%%%%%%%%%%%%%%%%%%%%%%%%%%%%%%%%%%%%%%%%%%%%%%%%%%%%%%%%%%%%%%%%%%%%%%%%%%%%%%%%%%

\begin{document}

\begin{center}
        \textbf{\Large Math 135, Calculus 1, Fall 2020}\\[10pt]
        {\large Written Homework 10-23}
\end{center}

\thispagestyle{empty}

\renewcommand{\thesection}{\Alph{section}}

\subsection*{Directions:}
Write your solutions neatly and clearly, and submit to Canvas.
In these problems, you should show all of your work in complete mathematical "sentences", writing complete English sentences when you explain your logic.
You are free (and encouraged!) to work with others, but make sure the solutions you write up your solutions indepedently.

\begin{exercise}[2 points]
        The equation
        \begin{equation}
                \label{DIFF_EQ}
                y'' = - A^2y
        \end{equation}
        is called a \textbf{differential equation}
        because it involves an unknown function $y$ and its derivatives $y'$ and $y''$.
        Prove that $y = \sin(Ax)$ satisfies Equation \cref{DIFF_EQ}.
        Find one additional solution (i.e. another function $y$ that satisfies Equation \cref{DIFF_EQ}).
\end{exercise}

\begin{exercise}[4 points]
        Fill in the blank with ``all'', ``no'', or ``some'' to make the following statements true.
        Note that ``some'' means one or more instances, but not all.

        \begin{itemize}
        \item If your answer is ``all'' or ``no'', give an example (graphically or algebraically) and a brief explaination as to why.
        \item If your answer is ``some'', then give two specific examples (graphically or algebraically) that illustrate why your answer is not ``all'' or ``no'', as well as a brief explaination.
        \end{itemize}

        \begin{enumerate}[(a)]
        \item For \underline{\qquad} functions $f$, if $f''(x) > 0$ on the interval $(a,b)$ then $f'(x) < 0$ on the interval $(a,b)$.
        \item For \underline{\qquad} functions $f$, the tangent line to $f(x)$ at $x = a$ will intersect the graph of $f(x)$ at exactly one point.
        \end{enumerate}
\end{exercise}

\begin{exercise}[4 points]
        A mayor announces that the town's expenses are increasing, but at a decreasing rate. Interpret this statement in terms of a function and its derivatives.
\end{exercise}
\end{document}


