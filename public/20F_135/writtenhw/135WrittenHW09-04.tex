\documentclass[11pt,reqno,final]{amsart}

\pdfcompresslevel=0
\pdfobjcompresslevel=0

\usepackage[dvipsnames]{xcolor}% adds colors
\usepackage{amsmath, amsthm}% {amsfonts, amssymb}

% New Characters
\usepackage[latin1]{inputenc}%
\usepackage[T1]{fontenc}

\usepackage{MnSymbol}
\usepackage[normalem]{ulem}% underlining

\usepackage[theoremfont, largesc]{newpxtext} % different text,math font
\usepackage{newpxmath}

\makeatletter
\DeclareMathRadical{\sqrtsign}{symbols}{112}{largesymbols}{112}
\let\sqrt=\undefined
\DeclareRobustCommand\sqrt{\@ifnextchar[\@sqrt{\mathpalette\@x@sqrt}}
\def\@x@sqrt#1#2{%
 \setbox\z@\hbox{$\m@th#1\sqrtsign{\mkern1mu #2}$}
 \mkern3mu\box\z@}
\makeatother




% Page Typesetting
\usepackage[final]{microtype}
\usepackage{relsize}
\usepackage[margin=1in]{geometry}
\usepackage{framed}
\usepackage{tikz}

\usepackage{hyperref}
\hypersetup{
  final,
  pdftitle={Math 135 - Written HW 1},
  pdfauthor={Bonventre}, 
  linktoc=page,
  pagebackref,
  colorlinks=true,
  citecolor=PineGreen,
  linkcolor=PineGreen,
  linkbordercolor=PineGreen,
}


% Internal References

\usepackage[inline,shortlabels]{enumitem}

\numberwithin{equation}{section} 
\numberwithin{figure}{section}

\usepackage[nameinlink,capitalise,noabbrev]{cleveref}

\crefname{equation}{}{} % get \cref to behave as \eqref

% \theoremstyle{plain} % bold name, italic text
\newtheorem{theorem}[equation]{Theorem}%
\newtheorem*{theorem*}{Theorem}%
\newtheorem{lemma}[equation]{Lemma}%
\newtheorem{proposition}[equation]{Proposition}%
\newtheorem{corollary}[equation]{Corollary}%
\newtheorem{conjecture}[equation]{Conjecture}%
\newtheorem*{conjecture*}{Conjecture}%
\newtheorem{claim}[equation]{Claim}%
\newtheorem{question}{Question}

\theoremstyle{definition} % bold name, plain text
\newtheorem{definition}[equation]{Definition}%
\newtheorem*{definition*}{Definition}%
\newtheorem{example}[equation]{Example}%
\newtheorem*{example*}{Example}%
\newtheorem{remark}[equation]{Remark}%
\newtheorem{notation}[equation]{Notation}%
\newtheorem{convention}[equation]{Convention}%
\newtheorem{assumption}[equation]{Assumption}%
\newtheorem{exercise}[question]{Exercise}

% ---------- macros
\newcommand{\set}[1]{\left\{#1\right\}}%
\newcommand{\sets}[2]{\left\{ #1 \;|\; #2\right\}}%
\newcommand{\longto}{\longrightarrow}%
\newcommand{\into}{\hookrightarrow}%
\newcommand{\onto}{\twoheadrightarrow}%

\usepackage{harpoon}
\newcommand{\vect}[1]{\text{\overrightharp{\ensuremath{#1}}}}

\newcommand{\del}{\partial}%

\newcommand{\ki}{\chi}
\newcommand{\ksi}{\xi}
\newcommand{\Ksi}{\Xi}

% %%%%%%%%%%%%%%%%%%%%%%%%%%%%%%%%%%%%%%%%%%%%%%%%%%%%%%%%%%%%%%%%%%%%%%%%%%%%%%%%%%%%%%%%%%%%%%%%%%%%

\begin{document}

\begin{center}
        \textbf{\Large Math 135, Calculus 1, Fall 2020}\\[10pt]
        {\large Written Homework 1 (Survey)}
\end{center}

\thispagestyle{empty}

\renewcommand{\thesection}{\Alph{section}}

This week's Written Homework serves two purposes:
\begin{enumerate*}[(a)]
\item you get to figure out how to submit assignments to Canvas, and
\item I get to learn a bit about you.
\end{enumerate*}
You can submit this to Canvas however you wish:
print, fill out, and scan;
fill out electronically;
or write answers on a blank piece of paper and scan.
However, Canvas will only accept PDF documents. There are apps to convert from pictures to PDFs, or take pictures as PDFs, if necessary.

Please feel free to email me ({\color{blue}pbonvent@holycross.edu}) if you have any questions.

\section{About You}

\begin{question}
        What is your name? How would you like to be addressed? What are your preferred pronouns?\\ $ $\\
\end{question}

\begin{question}
        What is your class year? Major? \\ $ $\\
\end{question}

\begin{question}
        What is your hometown? \\ $ $\\
\end{question}

\begin{question}
        What is your favorite book? Favorite Disney movie?\\ $ $\\
\end{question}

\begin{question}
        Anything else you would like me to know about you? \\ $ $\\
\end{question}

\section{You and Math}

\begin{question}
        What is your previous math experience? \vfill
\end{question}

\begin{question}
        Why are you taking this class? \vfill
\end{question}

\begin{question}
        What are your long-term math plans? \vfill
\end{question}

\begin{question}
        What worries you most about this class? \vfill
\end{question}

\newpage

\section{Distance Learning}

\begin{question}
        What is your experience with taking classes online?
        \vfill        
\end{question}

The sychronous portion of this course will be though the Zoom software.
\begin{question}
        How comfortable are you with using Zoom, or other online video conferencing/classroom systems?
        \vfill
\end{question}

If possible, I would like all students to use a tablet or laptop to attend class.
Additionally, in order to promote a collaborative environment for groupwork, I will be requesting that all students keep their videos \textbf{on} for the duration of the class.
\begin{question}
        How will you be accessing Zoom? \\ $ $\\
\end{question}

\begin{question}
        Do you have any concerns related to the technology/devices you will be using/asked to use to complete work in this course?Is this something you will be able to accommodate?
        (Potential complications could include: access to necessary technology, internet stability, lack of appropriate working space at home, etc).
        \vfill
\end{question}

\begin{question}
        What else do you want me to know about you so that I can support you to succeed this term?
        \vfill
\end{question}

I will strive to find a productive and amenable arrangment with each student that works for everyone.


\section{About This Class}

\begin{question}
        Please answer the following True/False questions:
        \begin{enumerate}[(a)]
        \item I will be lecturing in class every day \qquad \textbf{T} \quad / \quad \textbf{F}\\
        \item The majority of time in class will be spent with students quiet and me talking \qquad \textbf{T} \quad / \quad \textbf{F}\\
        \item Students are expected to prepare for class by completing an assignment beforehand \qquad \textbf{T} \quad / \quad \textbf{F}\\
        \item Students are encouraged to work together on WebAssign and Written Homework \qquad \textbf{T} \quad / \quad \textbf{F}\\
        \item Students are encouraged to work together on quizzes \qquad \textbf{T} \quad / \quad \textbf{F}
        \end{enumerate}
\end{question}

\end{document}


