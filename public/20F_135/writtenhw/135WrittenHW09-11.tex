\documentclass[11pt,reqno,final]{amsart}

\pdfcompresslevel=0
\pdfobjcompresslevel=0

\usepackage[dvipsnames]{xcolor}% adds colors
\usepackage{amsmath, amsthm}% {amsfonts, amssymb}

% New Characters
\usepackage[latin1]{inputenc}%
\usepackage[T1]{fontenc}

\usepackage{MnSymbol}
\usepackage[normalem]{ulem}% underlining

\usepackage[theoremfont, largesc]{newpxtext} % different text,math font
\usepackage{newpxmath}

\makeatletter
\DeclareMathRadical{\sqrtsign}{symbols}{112}{largesymbols}{112}
\let\sqrt=\undefined
\DeclareRobustCommand\sqrt{\@ifnextchar[\@sqrt{\mathpalette\@x@sqrt}}
\def\@x@sqrt#1#2{%
 \setbox\z@\hbox{$\m@th#1\sqrtsign{\mkern1mu #2}$}
 \mkern3mu\box\z@}
\makeatother




% Page Typesetting
\usepackage[final]{microtype}
\usepackage{relsize}
\usepackage[margin=1in]{geometry}
\usepackage{framed}
\usepackage{tikz}

\usepackage{hyperref}
\hypersetup{
  final,
  pdftitle={Math 135 - Written HW 1},
  pdfauthor={Bonventre}, 
  linktoc=page,
  pagebackref,
  colorlinks=true,
  citecolor=PineGreen,
  linkcolor=PineGreen,
  linkbordercolor=PineGreen,
}


% Internal References

\usepackage[inline,shortlabels]{enumitem}

\numberwithin{equation}{section} 
\numberwithin{figure}{section}

\usepackage[nameinlink,capitalise,noabbrev]{cleveref}

\crefname{equation}{}{} % get \cref to behave as \eqref

% \theoremstyle{plain} % bold name, italic text
\newtheorem{theorem}[equation]{Theorem}%
\newtheorem*{theorem*}{Theorem}%
\newtheorem{lemma}[equation]{Lemma}%
\newtheorem{proposition}[equation]{Proposition}%
\newtheorem{corollary}[equation]{Corollary}%
\newtheorem{conjecture}[equation]{Conjecture}%
\newtheorem*{conjecture*}{Conjecture}%
\newtheorem{claim}[equation]{Claim}%
\newtheorem{question}{Question}

\theoremstyle{definition} % bold name, plain text
\newtheorem{definition}[equation]{Definition}%
\newtheorem*{definition*}{Definition}%
\newtheorem{example}[equation]{Example}%
\newtheorem*{example*}{Example}%
\newtheorem{remark}[equation]{Remark}%
\newtheorem{notation}[equation]{Notation}%
\newtheorem{convention}[equation]{Convention}%
\newtheorem{assumption}[equation]{Assumption}%
\newtheorem{exercise}[question]{Exercise}

% ---------- macros
\newcommand{\set}[1]{\left\{#1\right\}}%
\newcommand{\sets}[2]{\left\{ #1 \;|\; #2\right\}}%
\newcommand{\longto}{\longrightarrow}%
\newcommand{\into}{\hookrightarrow}%
\newcommand{\onto}{\twoheadrightarrow}%

\usepackage{harpoon}
\newcommand{\vect}[1]{\text{\overrightharp{\ensuremath{#1}}}}

\newcommand{\del}{\partial}%

\newcommand{\ki}{\chi}
\newcommand{\ksi}{\xi}
\newcommand{\Ksi}{\Xi}

% %%%%%%%%%%%%%%%%%%%%%%%%%%%%%%%%%%%%%%%%%%%%%%%%%%%%%%%%%%%%%%%%%%%%%%%%%%%%%%%%%%%%%%%%%%%%%%%%%%%%

\begin{document}

\begin{center}
        \textbf{\Large Math 135, Calculus 1, Fall 2020}\\[10pt]
        {\large Written Homework 2: Due Friday September 11}
\end{center}

\thispagestyle{empty}

\renewcommand{\thesection}{\Alph{section}}

\subsection*{Directions:}
Write your solutions neatly and clearly, and submit to Canvas.
In these problems, you should show all of your work in complete mathematical "sentences", writing complete English sentences when you explain your logic.
You are free (and encouraged!) to work with others, but make sure the solutions you write up your solutions indepedently.

\begin{exercise}
        Fill in the black with ``all'', ``no'', or ``some'' to make the following statements true.
        \begin{itemize}
        \item If your answer is ``all'', explain why.
        \item If your answer is ``no'', given an example and explain.
        \item If you answer is ``some'', give two examples that demonstrate when the statement is true and when it is false. Explain your examples.
        \end{itemize}
        Note: an example must include either a graph or a specific function.
        \begin{enumerate}[(a)]
        \item For \underline{\qquad} real numbers $x$, $(x+2)^4 = x^4 + 16$.
        \item For \underline{\qquad} real numbers $x$, $\sqrt{x^4+8x^2+16} = x^2 + 4$.
        \item For \underline{\qquad} real numbers $x$, if $(x+2)(x-3) = 2$, then $x+2 = 2$ and $x-3 = 2$.
        \item For \underline{\qquad} functions $f$ and $g$, if $f$ and $g$ are both even functions, then $f+g$ is even.
        \item For \underline{\qquad} triples of real numbers $k$, $x$, and $y$, if $x<y$, then $kx < ky$.
        \end{enumerate}
\end{exercise}

$ $

\begin{exercise}
        An electric company charges its customers a fixed base charge of \$6 per month, plus 10 cents per kilowatt-hour (kWh) for the first 400 kWh, 11 cents per kWh for the next 500 kWh, and 15 cents for all additional kWh.
        \begin{enumerate}[(a)]
        \item Express the monthly cost $E$ as a function of the amount $x$ of electricity used.
        \item Graph the function $E$ for $0 \leq x \leq 2000$.
        \item Explain how your graph represents your function $E$.
        \end{enumerate}
\end{exercise}

\end{document}


