\documentclass[11pt,reqno,final]{amsart}

\pdfcompresslevel=0
\pdfobjcompresslevel=0

\usepackage[dvipsnames]{xcolor}% adds colors
\usepackage{amsmath, amsthm}% {amsfonts, amssymb}

% New Characters
\usepackage[latin1]{inputenc}%
\usepackage[T1]{fontenc}

\usepackage{MnSymbol}
\usepackage[normalem]{ulem}% underlining

\usepackage[theoremfont, largesc]{newpxtext} % different text,math font
\usepackage{newpxmath}

\makeatletter
\DeclareMathRadical{\sqrtsign}{symbols}{112}{largesymbols}{112}
% \let\sqrt=\undefined
% \DeclareRobustCommand\sqrt{\@ifnextchar[\@sqrt{\mathpalette\@x@sqrt}]}
% \def\@x@sqrt#1#2{%
%  \setbox\z@\hbox{$\m@th#1\sqrtsign{\mkern1mu #2}$}
%  \mkern3mu\box\z@}
\makeatother




% Page Typesetting
\usepackage[final]{microtype}
\usepackage{relsize}
\usepackage[margin=1in]{geometry}
\usepackage{framed}
\usepackage{tikz}

\usepackage{csquotes}

\usepackage{setspace}
\onehalfspacing

\usepackage{hyperref}
\hypersetup{
  final,
  pdftitle={Math 135 - Higher Derivatives and Trig Derivatives},
  pdfauthor={Bonventre}, 
  linktoc=page,
  pagebackref,
  colorlinks=true,
  citecolor=PineGreen,
  linkcolor=PineGreen,
  linkbordercolor=PineGreen,
}


% Internal References

\usepackage[inline,shortlabels]{enumitem}

% \numberwithin{equation}{section} 
\numberwithin{figure}{section}

\usepackage[nameinlink,capitalise,noabbrev]{cleveref}

\crefname{equation}{}{} % get \cref to behave as \eqref

% \theoremstyle{plain} % bold name, italic text
\newtheorem{theorem}[equation]{Theorem}%
\newtheorem*{theorem*}{Theorem}%
\newtheorem{lemma}[equation]{Lemma}%
\newtheorem{proposition}[equation]{Proposition}%
\newtheorem{corollary}[equation]{Corollary}%
\newtheorem{conjecture}[equation]{Conjecture}%
\newtheorem*{conjecture*}{Conjecture}%
\newtheorem{claim}[equation]{Claim}%
\newtheorem{question}{Question}

\theoremstyle{definition} % bold name, plain text
\newtheorem{definition}[equation]{Definition}%
\newtheorem*{definition*}{Definition}%
\newtheorem{example}[equation]{Example}%
\newtheorem*{example*}{Example}%
\newtheorem{remark}[equation]{Remark}%
\newtheorem{notation}[equation]{Notation}%
\newtheorem{convention}[equation]{Convention}%
\newtheorem{assumption}[equation]{Assumption}%
\newtheorem{exercise}[question]{Exercise}

% ---------- macros
\newcommand{\set}[1]{\left\{#1\right\}}%
\newcommand{\sets}[2]{\left\{ #1 \;|\; #2\right\}}%
\newcommand{\longto}{\longrightarrow}%
\newcommand{\into}{\hookrightarrow}%
\newcommand{\onto}{\twoheadrightarrow}%

\usepackage{harpoon}
\newcommand{\vect}[1]{\text{\overrightharp{\ensuremath{#1}}}}

\newcommand{\del}{\partial}%

\newcommand{\ki}{\chi}
\newcommand{\ksi}{\xi}
\newcommand{\Ksi}{\Xi}

\newcommand{\dlim}{\displaystyle\lim}

% %%%%%%%%%%%%%%%%%%%%%%%%%%%%%%%%%%%%%%%%%%%%%%%%%%%%%%%%%%%%%%%%%%%%%%%%%%%%%%%%%%%%%%%%%%%%%%%%%%%%

\begin{document}


\begin{center}
        \textbf{\Large Math 135, Calculus 1, Fall 2020}\\[10pt]
        {\large 10-16: Higher Order Derivatives (Section 3.5) and Trig Derivatives (Section 3.6)}
\end{center}

\thispagestyle{empty}


\renewcommand{\thesection}{\Alph{section}}

% \vspace{-1pt}

Last week, we introduced the \textbf{derivative function} $f'(x)$ of a function $f(x)$, whose evaluation $f'(a)$ at the point $x=a$ is give by:
\begin{itemize}
\item the slope of the tangent line at $x=a$
\item the instantaneous velocity at time $x = a$
\item the instantaneous rate of change of $f$ with respect to $x$
\end{itemize}

Today, we'll be discussing \textbf{higher order derivatives} and the derivatives of \textbf{trig functions}.

\section{Higher Derivatives}

The derivative $f'(x)$ of a function $f(x)$ gives the \textbf{slope} of the function $f$ at the point $x$.
However, $f'(x)$ is also a function, and so we can ask: ``What is the derivative of the derivative?''

\begin{definition}
        The \textbf{second derivative} of $f(x)$ is the function
        \begin{framed}
                \[
                        f''(x) = \dfrac{d}{dx}\big( f'(x) \big) = \dfrac{d^2f}{dx^2} = \dfrac{d^2y}{dx^2}.                
                \]
        \end{framed}
\end{definition}
To compute the second derivative, we simply take the derivative twice.

\begin{exercise}
        Suppose that $f(x) = 2x^4-3x+e^x$. Find $f'(x)$ and $f''(x)$.
        \vfill
\end{exercise}

\subsection{The second derivative and concavity}

The second derivative measures the change in $f'(x)$, i.e. the change in the slope of $f(x)$.
What does this really mean?

Recall:
\begin{itemize}
\item $\dfrac{d}{dx}(f)|_{x=a} > 0 \quad \Leftrightarrow \quad$
        the slope is positive at $x=a$ $\quad \Leftrightarrow \quad$
        $f(x)$ is \textbf{increasing} at $x=a$
\item $\dfrac{d}{dx}(f)|_{x=a} < 0 \quad \Leftrightarrow \quad$
        the slope is negative at $x=a$ $\quad \Leftrightarrow \quad$
        $f(x)$ is \textbf{decreasing} at $x=a$
\end{itemize}

If $f''(a) >0$, then $\dfrac{d}{dx}(f')|_{x=a}>0$, so the slopes of $f$ are increasing at $x=a$.
There are two options:
\begin{itemize}
\item If the slopes are already positive, then they are getting bigger, so the curve is getting steeper, increasing at a faster rate (like $e^x$)
\item If the slopes are negative, then the function $f(x)$ is still decreasing, but beginning to flatten out: the negative slopes are increasing towards (and possibly past) zero.
\end{itemize}
In these cases, we say the graph of $f$ is \textbf{concave up} at $x=a$.

Similarly, the reverse options can happen if $f''(a)<0$, and the graph is \textbf{concave down}.

\newpage

\begin{exercise}
        Sketch the graph of a function $g$ such that $g'(x) < 0$ and $g''(x) > 0$ everywhere.
        \vfill
\end{exercise}

\begin{exercise}
        Former President Nixon famously said,
        ``Although the rate of inflation is increasing, it is increasing at a decreasing rate.''
        Let $r(t)$ denote the rate of inflation.
        According to President Nixon, what are the signs ($+$, $-$, or 0)
        of $r'(t)$ and $r''(t)$?\\[10pt]        
\end{exercise}

\subsection{Higher Derivatives}.

In good cases, we can continue to take derivatives of derivatives.
\begin{itemize}
\item We write $f'''(x)$ for the \textbf{third derivative} $\dfrac{d}{dx}(f''(x))$.
\item More generally, we write $f^{(n)}(x)$ for the \textbf{$n$-th derivative} of $f(x)$.
\end{itemize}

\begin{exercise}
        Suppose that $f(x) = xe^x$.
        \begin{enumerate}[(a)]
        \item Find and simplify $f'(x)$, $f''(x)$, and $f'''(x)$.
                \vfill
        \item Find a general formula, in terms of $n$, of the $n$-th derivative $f^{(n)}(x)$.
                \vfill
        \end{enumerate}
\end{exercise}


\begin{exercise}
        The inflation rate is given by the (positive) rate of change of the \textbf{consumer price index}.
        Let $p(t)$ denote the consumer price index.
        According to Nixon, what are the signs ($+$, $-$, or 0) of $p'(t)$, $p''(t)$, and $p'''(t)$?\\[10pt]
\end{exercise}

\begin{exercise}
        A news report out of Massachusetts yesterday said:
        \begin{displayquote}
                The total number of COVID cases that were confirmed last week grew to 4,560 today.
                That's a 12\% increase over the previous week
                and an 83\% increase in cases over the week of Sept. 13, when cases began to rise at a higher rate.        
        \end{displayquote}
        Let $N(t)$ denote the cumulative total number of cases of COVID-19 in Massachusetts.
        What derivative of $N$ went from negative to positive on September 13?
        Using evidence from the news article, is that derivative still positive?
        \vfill
\end{exercise}

\newpage

\section{Trig Derivatives}

Using the trig identities $\sin(A+B) = \sin A \cdot \cos B + \cos A \sin B$,
along with the trig limits $\dlim_{x \to 0}\dfrac{\sin x}{x} = 1$ and $\dlim_{x \to 0}\dfrac{1 - \cos x}{x} = 0$,
we can compute the derivatives of $\sin(x)$ and $\cos(x)$:

\begin{theorem}
        If $x$ is measured in radians, then
        \begin{framed}
                \[
                        \dfrac{d}{dx}(\sin x) = \cos x,
                        \qquad \qquad
                        \dfrac{d}{dx}(\cos x) = -\sin x.
                        \]
        \end{framed}
\end{theorem}

To prove the first formula, let $f(x) = \sin x$. Then
\begin{align*}
  f'(x) &= \dlim_{h \to 0} \dfrac{\sin(x+h) - \sin(x)}{h}\\
        &= \dlim_{h \to 0} \dfrac{\sin x \cdot \cos h + \cos x \cdot \sin h - \sin x}{h}\\
        &= \dlim_{h \to 0} \dfrac{ \sin x \cdot (\cos h - 1) + \cos x \cdot \sin h}{h}\\
        &= \dlim_{h \to 0} \sin x \cdot \dfrac{\cos h - 1}{h} \quad + \quad \dlim_{h \to 0} \cos x \cdot \dfrac{\sin h}{h}\\
        &= \sin x \cdot 0 + \cos x \cdot 1\\
        &= \cos x.          
\end{align*}

\begin{exercise}
        Use the quotient rule and the above results to prove that $\dfrac{d}{dx}(\tan x) = \sec^2 x$.
        \vfill
\end{exercise}

\begin{exercise}
        Show that $\dfrac{d}{dx}(\sec x) = \sec x \cdot \tan x$.
        \vfill
\end{exercise}

\begin{exercise}
        If $g(x) = x^3 \sin x$, find a simplify $g'(x)$ and $g''(x)$.
        \vfill
\end{exercise}

\end{document}
