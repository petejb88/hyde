\documentclass[11pt,reqno,final]{amsart}

\pdfcompresslevel=0
\pdfobjcompresslevel=0

\usepackage[dvipsnames]{xcolor}% adds colors
\usepackage{amsmath, amsthm}% {amsfonts, amssymb}

% New Characters
\usepackage[latin1]{inputenc}%
\usepackage[T1]{fontenc}

\usepackage{MnSymbol}
\usepackage[normalem]{ulem}% underlining

\usepackage[theoremfont, largesc]{newpxtext} % different text,math font
\usepackage{newpxmath}

\makeatletter
\DeclareMathRadical{\sqrtsign}{symbols}{112}{largesymbols}{112}
% \let\sqrt=\undefined
% \DeclareRobustCommand\sqrt{\@ifnextchar[\@sqrt{\mathpalette\@x@sqrt}]}
% \def\@x@sqrt#1#2{%
%  \setbox\z@\hbox{$\m@th#1\sqrtsign{\mkern1mu #2}$}
%  \mkern3mu\box\z@}
\makeatother




% Page Typesetting
\usepackage[final]{microtype}
\usepackage{relsize}
\usepackage[margin=1in]{geometry}
\usepackage{framed}
\usepackage{tikz}

\usepackage{csquotes}

\usepackage{setspace}
\onehalfspacing

\usepackage{hyperref}
\hypersetup{
  final,
  pdftitle={Math 135 - L'H\^{o}pital's Rule},
  pdfauthor={Bonventre}, 
  linktoc=page,
  pagebackref,
  colorlinks=true,
  citecolor=PineGreen,
  linkcolor=PineGreen,
  linkbordercolor=PineGreen,
}


% Internal References

\usepackage[inline,shortlabels]{enumitem}

% \numberwithin{equation}{section} 
\numberwithin{figure}{section}

\usepackage[nameinlink,capitalise,noabbrev]{cleveref}

\crefname{equation}{}{} % get \cref to behave as \eqref

% \theoremstyle{plain} % bold name, italic text
\newtheorem{theorem}[equation]{Theorem}%
\newtheorem*{theorem*}{Theorem}%
\newtheorem{lemma}[equation]{Lemma}%
\newtheorem{proposition}[equation]{Proposition}%
\newtheorem{corollary}[equation]{Corollary}%
\newtheorem{conjecture}[equation]{Conjecture}%
\newtheorem*{conjecture*}{Conjecture}%
\newtheorem{claim}[equation]{Claim}%
\newtheorem{question}{Question}

\theoremstyle{definition} % bold name, plain text
\newtheorem{definition}[equation]{Definition}%
\newtheorem*{definition*}{Definition}%
\newtheorem{example}[equation]{Example}%
\newtheorem*{example*}{Example}%
\newtheorem{remark}[equation]{Remark}%
\newtheorem{notation}[equation]{Notation}%
\newtheorem{convention}[equation]{Convention}%
\newtheorem{assumption}[equation]{Assumption}%
\newtheorem{exercise}[question]{Exercise}

% ---------- macros
\newcommand{\set}[1]{\left\{#1\right\}}%
\newcommand{\sets}[2]{\left\{ #1 \;|\; #2\right\}}%
\newcommand{\longto}{\longrightarrow}%
\newcommand{\into}{\hookrightarrow}%
\newcommand{\onto}{\twoheadrightarrow}%

\usepackage{harpoon}
\newcommand{\vect}[1]{\text{\overrightharp{\ensuremath{#1}}}}

\newcommand{\del}{\partial}%

\newcommand{\ki}{\chi}
\newcommand{\ksi}{\xi}
\newcommand{\Ksi}{\Xi}

\newcommand{\dlim}{\displaystyle\lim}

% %%%%%%%%%%%%%%%%%%%%%%%%%%%%%%%%%%%%%%%%%%%%%%%%%%%%%%%%%%%%%%%%%%%%%%%%%%%%%%%%%%%%%%%%%%%%%%%%%%%%

\begin{document}


\begin{center}
        \textbf{\Large Math 135, Calculus 1, Fall 2020}\\[10pt]
        {\large 11-30: L'H\^{o}pital's Rule}
\end{center}

\thispagestyle{empty}


\renewcommand{\thesection}{\Alph{section}}

% \vspace{-1pt}

The \textbf{derivative} $f'(x)$ of a function $y=f(x)$ gives:
\begin{itemize}
\item the slope of the tangent line
\item the instantaneous rate of change of $y$ with respect to $x$
\end{itemize}

\subsection*{Common Problem} Suppose we want to compute the limit of a rational function
\[
        \dlim_{x \to a} \dfrac{f(x)}{g(x)}
\]
but when we plug in $x = a$ (including $a = \pm \infty$), we get one of the following \textbf{indeterminant forms}:
\[
        \dfrac{0}{0} \qquad \mbox{or} \qquad  \dfrac{\pm \infty}{\infty}
\]
In this case, we can apply L'H\^{o}pital's Rule to help compute this limit.\\
\begin{theorem}[L'H\^{o}pital's Rule, Guillaume Fran\c{c}ois Antoine Marquis de L'H\^{o}pital, 1696]
        If $f$ and $g$ are differentiable functions such that
        $\frac{f(a)}{g(a)}$
        is one of the above indeterminant forms, then
        \begin{framed}
                \[
                        \dlim_{x \to a} \dfrac{f(x)}{g(x)} = \dlim_{x \to a}\dfrac{f'(x)}{g'(x)}.
                \]
        \end{framed}
\end{theorem}

\begin{itemize}
\item You may have to apply the rule multiple times to determine the limit
\item The rule was actually discovered by Bernoulli in 1964.\\
\end{itemize}

\begin{example}
        Consider $\dlim_{x \to 0} \dfrac{\sin x}{x}$, where $x$ is in radians.
        Plugging in $x = 0$ yields the indeterminant form $\dfrac{0}{0}$, so L'H\^opital's Rule applies.
        Thus we have
        \[
                \dlim_{x \to 0} \dfrac{\sin x}{x} = \dlim_{x \to 0} \dfrac{\cos x}{1} = \dfrac{\cos 0}{1} = 1.
        \]
        This matches our calculation from Section 2.6.\\
\end{example}

\begin{exercise}
        Use L'H\^opital's Rule to compute the other important trig limit\\
        $ $\\
        $\dlim_{x \to 0} \dfrac{1 - \cos x}{x} = $
         
\end{exercise}

\newpage

\begin{exercise}
        Ccompute the following limits.
        Make sure to \textbf{first check} that the rule actually applies.
        \begin{enumerate}[(a)]
        \item $\dlim_{x \to 2} \dfrac{x^3 - 8}{x^4 - x^3 - 8}$
                \vfill
        \item $\dlim_{x \to 0} \dfrac{1 - \cos x}{x^2}$
                \vfill
        \item $\dlim_{x \to 3} \dfrac{x - 3}{x}$
                \vfill
        \item $\dlim_{x \to \infty} \dfrac{ \ln x}{\sqrt{x}}$
                \vfill
        \item $\dlim_{x \to \infty} \dfrac{e^x}{x}$
                \vfill
        \end{enumerate}
\end{exercise}

\begin{exercise}
        Use L'H\^opital's Rule to find any horizontal asymptotes of the following functions.
        \begin{enumerate}[(a)]
        \item $f(x) = \dfrac{8x^3 - 4x+\pi}{-3x^3 + 7x^2 + 5}$
                \vfill
        \item $g(x) = \dfrac{3e^{2x} + 5x}{e^{2x} + 8x}$
                \vfill
        \end{enumerate}
\end{exercise}

\end{document}
