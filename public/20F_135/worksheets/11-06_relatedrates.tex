\documentclass[11pt,reqno,final]{amsart}

\pdfcompresslevel=0
\pdfobjcompresslevel=0

\usepackage[dvipsnames]{xcolor}% adds colors
\usepackage{amsmath, amsthm}% {amsfonts, amssymb}

% New Characters
\usepackage[latin1]{inputenc}%
\usepackage[T1]{fontenc}

\usepackage{MnSymbol}
\usepackage[normalem]{ulem}% underlining

\usepackage[theoremfont, largesc]{newpxtext} % different text,math font
\usepackage{newpxmath}

\makeatletter
\DeclareMathRadical{\sqrtsign}{symbols}{112}{largesymbols}{112}
% \let\sqrt=\undefined
% \DeclareRobustCommand\sqrt{\@ifnextchar[\@sqrt{\mathpalette\@x@sqrt}]}
% \def\@x@sqrt#1#2{%
%  \setbox\z@\hbox{$\m@th#1\sqrtsign{\mkern1mu #2}$}
%  \mkern3mu\box\z@}
\makeatother




% Page Typesetting
\usepackage[final]{microtype}
\usepackage{relsize}
\usepackage[margin=1in]{geometry}
\usepackage{framed}
\usepackage{tikz}

\usepackage{csquotes}

\usepackage{setspace}
\onehalfspacing

\usepackage{hyperref}
\hypersetup{
  final,
  pdftitle={Math 135 - Related Rates},
  pdfauthor={Bonventre}, 
  linktoc=page,
  pagebackref,
  colorlinks=true,
  citecolor=PineGreen,
  linkcolor=PineGreen,
  linkbordercolor=PineGreen,
}


% Internal References

\usepackage[inline,shortlabels]{enumitem}

% \numberwithin{equation}{section} 
\numberwithin{figure}{section}

\usepackage[nameinlink,capitalise,noabbrev]{cleveref}

\crefname{equation}{}{} % get \cref to behave as \eqref

% \theoremstyle{plain} % bold name, italic text
\newtheorem{theorem}[equation]{Theorem}%
\newtheorem*{theorem*}{Theorem}%
\newtheorem{lemma}[equation]{Lemma}%
\newtheorem{proposition}[equation]{Proposition}%
\newtheorem{corollary}[equation]{Corollary}%
\newtheorem{conjecture}[equation]{Conjecture}%
\newtheorem*{conjecture*}{Conjecture}%
\newtheorem{claim}[equation]{Claim}%
\newtheorem{question}{Question}

\theoremstyle{definition} % bold name, plain text
\newtheorem{definition}[equation]{Definition}%
\newtheorem*{definition*}{Definition}%
\newtheorem{example}[equation]{Example}%
\newtheorem*{example*}{Example}%
\newtheorem{remark}[equation]{Remark}%
\newtheorem{notation}[equation]{Notation}%
\newtheorem{convention}[equation]{Convention}%
\newtheorem{assumption}[equation]{Assumption}%
\newtheorem{exercise}[question]{Exercise}

% ---------- macros
\newcommand{\set}[1]{\left\{#1\right\}}%
\newcommand{\sets}[2]{\left\{ #1 \;|\; #2\right\}}%
\newcommand{\longto}{\longrightarrow}%
\newcommand{\into}{\hookrightarrow}%
\newcommand{\onto}{\twoheadrightarrow}%

\usepackage{harpoon}
\newcommand{\vect}[1]{\text{\overrightharp{\ensuremath{#1}}}}

\newcommand{\del}{\partial}%

\newcommand{\ki}{\chi}
\newcommand{\ksi}{\xi}
\newcommand{\Ksi}{\Xi}

\newcommand{\dlim}{\displaystyle\lim}

% %%%%%%%%%%%%%%%%%%%%%%%%%%%%%%%%%%%%%%%%%%%%%%%%%%%%%%%%%%%%%%%%%%%%%%%%%%%%%%%%%%%%%%%%%%%%%%%%%%%%

\begin{document}


\begin{center}
        \textbf{\Large Math 135, Calculus 1, Fall 2020}\\[10pt]
        {\large 11-06:  Related Rates (Section 3.10)}
\end{center}

\thispagestyle{empty}


\renewcommand{\thesection}{\Alph{section}}

% \vspace{-1pt}

The \textbf{derivative} $f'(x)$ of a function $y=f(x)$ gives:
\begin{itemize}
\item the slope of the tangent line
\item the instantaneous rate of change of $y$ with respect to $x$
\end{itemize}


\section{Related Rates}

If two quantities are:
\begin{enumerate*}[(i)]
\item related by some equation, and
\item are \textbf{both} changing with respect to another quantity,
\end{enumerate*}
then we can use the given equation and the chain rule to 'relate their rates''.


Let's go through an example together:
\begin{example}      
        Suppose that each side of a square is increasing at a constant rate of 10 $\mathrm{cm}^2/\mathrm{sec}$.
        At what rate is the area increasing when the area of the square is 36 cm$^2$?
        \begin{enumerate}[(a)]
        \item \textbf{Diagram:} Draw a picture of the square. Label the variables of interest.
                \vfill
        \item \textbf{Rates:} What rate of change is given to us in the statement of the problem?
                What rate of change are we asked to compute?
                \vfill
        \item \textbf{Equation:} Write an equation from geometry that relates the variables $A$ and $x$.
                \vfill
        \item \textbf{Differentiate:} Both the area and the side lengths are increasing as a function of time.
                Use implicit differentiation to take the derivative of the formula from Part (c) with respect to time $t$.
                \vfill
        \item \textbf{Substitute and Solve:} Plug all known quantities into your quation from Part (d),
                and solve for the desired rate. Answer the question asked.
                \vfill
        \end{enumerate}
\end{example}

\newpage


\begin{exercise}
        A spherical snowball melts in such a way that the instant at which its radius is 20cm, its radius is decreasing at 3 cm/min.
        At what rate is the volume of the ball of snow changing at that instant?
        \begin{enumerate}[(a)]
        \item \textbf{Diagram:} Draw a picture of the melting snowball. Label the variables of interest.\\
                \textit{(Hint: the statement of the problem mentions three variables. What are they?)}
                \vfill
                \vfill
        \item \textbf{Rates:} What rate of change is given to us in the statement of the problem?
                What rate of change are we asked to compute?
                Include units.
                \vfill
        \item \textbf{Equation:} The rates in the previous part involved the variables $V$ and $r$.
                Write an equation from geometry relating $V$ and $r$.
                \vfill
        \item \textbf{Differentiate:} Because the snowball is melting, both the radius and volume are functions of time.
                Use implicit differentiation to take the derivative of your formula from Part (c) with respect to time $t$ in minutes.
                \vfill
        \item \textbf{Substitute and Solve:} Plug all known quantities into your equation from Part (d) and solve for the desired rate.
                Answer the question asked.
                \vfill
        \end{enumerate}
\end{exercise}

\newpage


\begin{exercise}
        Omar flies his kite 150m high, where the wind causes it to move horizontally away from him at the rate of 5m per second.
        In order to maintain the kite at a height of 150m, Omar must allow more string to be let out.
        At what rate is the string being let out when the length of the string already out is 250m?
        \begin{enumerate}[(a)]
        \item \textbf{Diagram:}
                \vfill
                \vfill
        \item \textbf{Rates:}
                \vfill
        \item \textbf{Equation:}
                \vfill
        \item \textbf{Differentiate:}
                \vfill
        \item \textbf{Substitute and Solve:}
                \vfill
        \end{enumerate}
\end{exercise}

\newpage


\begin{exercise}
        On the shore sits Sea Lion Rock.
        A lighthouse stands off-shore, 100 yards east of Sea Lion Rock.
        173 yards due north of Sea Lion Rock is the exclusive Sea Lion Motel.
        The lighthouse light rotates twice a minute. At the moment the beam of light hits the motel, how fast is the beam of light moving along the coast?
        % \begin{enumerate}[(a)]
        % \item \textbf{Diagram:}
        %         \vfill
        %         \vfill
        % \item \textbf{Rates:}
        %         \vfill
        % \item \textbf{Equation:}
        %         \vfill
        % \item \textbf{Differentiate:}
        %         \vfill
        % \item \textbf{Substitute and Solve:}
        %         \vfill
        % \end{enumerate}
\end{exercise}


\end{document}
