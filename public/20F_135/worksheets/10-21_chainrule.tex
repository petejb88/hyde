\documentclass[11pt,reqno,final]{amsart}

\pdfcompresslevel=0
\pdfobjcompresslevel=0

\usepackage[dvipsnames]{xcolor}% adds colors
\usepackage{amsmath, amsthm}% {amsfonts, amssymb}

% New Characters
\usepackage[latin1]{inputenc}%
\usepackage[T1]{fontenc}

\usepackage{MnSymbol}
\usepackage[normalem]{ulem}% underlining

\usepackage[theoremfont, largesc]{newpxtext} % different text,math font
\usepackage{newpxmath}

\makeatletter
\DeclareMathRadical{\sqrtsign}{symbols}{112}{largesymbols}{112}
% \let\sqrt=\undefined
% \DeclareRobustCommand\sqrt{\@ifnextchar[\@sqrt{\mathpalette\@x@sqrt}]}
% \def\@x@sqrt#1#2{%
%  \setbox\z@\hbox{$\m@th#1\sqrtsign{\mkern1mu #2}$}
%  \mkern3mu\box\z@}
\makeatother




% Page Typesetting
\usepackage[final]{microtype}
\usepackage{relsize}
\usepackage[margin=1in]{geometry}
\usepackage{framed}
\usepackage{tikz}

\usepackage{csquotes}

\usepackage{setspace}
\onehalfspacing

\usepackage{hyperref}
\hypersetup{
  final,
  pdftitle={Math 135 - Chain Rule},
  pdfauthor={Bonventre}, 
  linktoc=page,
  pagebackref,
  colorlinks=true,
  citecolor=PineGreen,
  linkcolor=PineGreen,
  linkbordercolor=PineGreen,
}


% Internal References

\usepackage[inline,shortlabels]{enumitem}

% \numberwithin{equation}{section} 
\numberwithin{figure}{section}

\usepackage[nameinlink,capitalise,noabbrev]{cleveref}

\crefname{equation}{}{} % get \cref to behave as \eqref

% \theoremstyle{plain} % bold name, italic text
\newtheorem{theorem}[equation]{Theorem}%
\newtheorem*{theorem*}{Theorem}%
\newtheorem{lemma}[equation]{Lemma}%
\newtheorem{proposition}[equation]{Proposition}%
\newtheorem{corollary}[equation]{Corollary}%
\newtheorem{conjecture}[equation]{Conjecture}%
\newtheorem*{conjecture*}{Conjecture}%
\newtheorem{claim}[equation]{Claim}%
\newtheorem{question}{Question}

\theoremstyle{definition} % bold name, plain text
\newtheorem{definition}[equation]{Definition}%
\newtheorem*{definition*}{Definition}%
\newtheorem{example}[equation]{Example}%
\newtheorem*{example*}{Example}%
\newtheorem{remark}[equation]{Remark}%
\newtheorem{notation}[equation]{Notation}%
\newtheorem{convention}[equation]{Convention}%
\newtheorem{assumption}[equation]{Assumption}%
\newtheorem{exercise}[question]{Exercise}

% ---------- macros
\newcommand{\set}[1]{\left\{#1\right\}}%
\newcommand{\sets}[2]{\left\{ #1 \;|\; #2\right\}}%
\newcommand{\longto}{\longrightarrow}%
\newcommand{\into}{\hookrightarrow}%
\newcommand{\onto}{\twoheadrightarrow}%

\usepackage{harpoon}
\newcommand{\vect}[1]{\text{\overrightharp{\ensuremath{#1}}}}

\newcommand{\del}{\partial}%

\newcommand{\ki}{\chi}
\newcommand{\ksi}{\xi}
\newcommand{\Ksi}{\Xi}

\newcommand{\dlim}{\displaystyle\lim}

% %%%%%%%%%%%%%%%%%%%%%%%%%%%%%%%%%%%%%%%%%%%%%%%%%%%%%%%%%%%%%%%%%%%%%%%%%%%%%%%%%%%%%%%%%%%%%%%%%%%%

\begin{document}


\begin{center}
        \textbf{\Large Math 135, Calculus 1, Fall 2020}\\[10pt]
        {\large 10-21: The Chain Rule, Part II (Section 3.7)}
\end{center}

\thispagestyle{empty}


\renewcommand{\thesection}{\Alph{section}}

% \vspace{-1pt}

Last week, we introduced the \textbf{derivative function} $f'(x)$ of a function $f(x)$, whose evaluation $f'(a)$ at the point $x=a$ is give by:
\begin{itemize}
\item the slope of the tangent line at $x=a$
\item the instantaneous velocity at time $x = a$
\item the instantaneous rate of change of $f$ with respect to $x$
\end{itemize}

Today: More with the chain rule.
\section{Chain Rule}

The chain rule gives us a way to compute the derivative of a \textbf{composite} of two functions.
\begin{theorem}
        If $O(x)$ and $I(x)$ are differentiable functions, then so is the composite $O(I(x)) = (O \circ I)(x)$.
        Moreover,
        \begin{framed}
                \[
                        \dfrac{d}{dx}\Big( O(I(x)) \Big) = O'\big( I(x) \big) \cdot I'(x).
                \]
        \end{framed}
\end{theorem}

\begin{exercise}
        Compute the derivative of $f(x) = \sin(x^2)$.
        \vfill
\end{exercise}

\begin{exercise}
        Suppose that $h(x) = f(g(x))$, and that $f'(3) = 4$, $f(3) = 2$, $f'(6) = -1$, $g(3) = 6$, $g'(3) = 7$, $g(2) = 4$, and $g'(2) = 11$.
        Find $h'(3)$.
        \vfill
\end{exercise}

\newpage

\subsection*{Leibniz Notation}
If $y = f(x)$ and $x = g(t)$, then $y$ is a function of $t$ by $y = f(g(t))$.
By the Chain Rule, we have that
$\frac{dy}{dt} = f'(g(t)) \cdot g'(t)$, or more succinctly,
\begin{framed}
        \[
                \dfrac{dy}{dt} = \dfrac{dy}{dx} \cdot \dfrac{dx}{dt}.
        \]
\end{framed}

When computing the derivative of the inside function, you may need to use any of the derivative rules we have discussed so far:
the product rule, quotient rule, or even the chain rule itself.

\begin{exercise}
        Compute $\dfrac{dy}{dt}$ when $y = \sqrt{\dfrac{t+1}{t-1}}$.
        \vfill
\end{exercise}

\begin{exercise}
        Compute $\dfrac{d}{dt} \Big( \tan(\cos(e^{6t})) \Big)$.
        \vfill
\end{exercise}

\begin{exercise}
        Find and simplify $F'(x)$ if $F(x) = \sin(x^2)\cos(x^2)$.
        \vfill
\end{exercise}
        
\end{document}
