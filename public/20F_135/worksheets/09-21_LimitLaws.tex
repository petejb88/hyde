\documentclass[11pt,reqno,final]{amsart}

\pdfcompresslevel=0
\pdfobjcompresslevel=0

\usepackage[dvipsnames]{xcolor}% adds colors
\usepackage{amsmath, amsthm}% {amsfonts, amssymb}

% New Characters
\usepackage[latin1]{inputenc}%
\usepackage[T1]{fontenc}

\usepackage{MnSymbol}
\usepackage[normalem]{ulem}% underlining

\usepackage[theoremfont, largesc]{newpxtext} % different text,math font
\usepackage{newpxmath}

\makeatletter
\DeclareMathRadical{\sqrtsign}{symbols}{112}{largesymbols}{112}
\let\sqrt=\undefined
\DeclareRobustCommand\sqrt{\@ifnextchar[\@sqrt{\mathpalette\@x@sqrt}]}
\def\@x@sqrt#1#2{%
 \setbox\z@\hbox{$\m@th#1\sqrtsign{\mkern1mu #2}$}
 \mkern3mu\box\z@}
\makeatother




% Page Typesetting
\usepackage[final]{microtype}
\usepackage{relsize}
\usepackage[margin=1in]{geometry}
\usepackage{framed}
\usepackage{tikz}
\usepackage{setspace}

\usepackage{hyperref}
\hypersetup{
  final,
  pdftitle={Math 135 - Limits},
  pdfauthor={Bonventre}, 
  linktoc=page,
  pagebackref,
  colorlinks=true,
  citecolor=PineGreen,
  linkcolor=PineGreen,
  linkbordercolor=PineGreen,
}


% Internal References

\usepackage[inline,shortlabels]{enumitem}

\numberwithin{equation}{section} 
\numberwithin{figure}{section}

\usepackage[nameinlink,capitalise,noabbrev]{cleveref}

\crefname{equation}{}{} % get \cref to behave as \eqref

% \theoremstyle{plain} % bold name, italic text
\newtheorem{theorem}[equation]{Theorem}%
\newtheorem*{theorem*}{Theorem}%
\newtheorem{lemma}[equation]{Lemma}%
\newtheorem{proposition}[equation]{Proposition}%
\newtheorem{corollary}[equation]{Corollary}%
\newtheorem{conjecture}[equation]{Conjecture}%
\newtheorem*{conjecture*}{Conjecture}%
\newtheorem{claim}[equation]{Claim}%
\newtheorem{question}{Question}

\theoremstyle{definition} % bold name, plain text
\newtheorem{definition}[equation]{Definition}%
\newtheorem*{definition*}{Definition}%
\newtheorem{example}[equation]{Example}%
\newtheorem*{example*}{Example}%
\newtheorem{remark}[equation]{Remark}%
\newtheorem{notation}[equation]{Notation}%
\newtheorem{convention}[equation]{Convention}%
\newtheorem{assumption}[equation]{Assumption}%
\newtheorem{exercise}[question]{Exercise}

% ---------- macros
\newcommand{\set}[1]{\left\{#1\right\}}%
\newcommand{\sets}[2]{\left\{ #1 \;|\; #2\right\}}%
\newcommand{\longto}{\longrightarrow}%
\newcommand{\into}{\hookrightarrow}%
\newcommand{\onto}{\twoheadrightarrow}%

\usepackage{harpoon}
\newcommand{\vect}[1]{\text{\overrightharp{\ensuremath{#1}}}}

\newcommand{\del}{\partial}%

\newcommand{\ki}{\chi}
\newcommand{\ksi}{\xi}
\newcommand{\Ksi}{\Xi}

% %%%%%%%%%%%%%%%%%%%%%%%%%%%%%%%%%%%%%%%%%%%%%%%%%%%%%%%%%%%%%%%%%%%%%%%%%%%%%%%%%%%%%%%%%%%%%%%%%%%%

\begin{document}
\onehalfspacing

\begin{center}
        \textbf{\Large Math 135, Calculus 1, Fall 2020}\\[10pt]
        {\large 09-21: Limit Law Guide}
\end{center}

\thispagestyle{empty}

\renewcommand{\thesection}{\Alph{section}}

Suppose that $\displaystyle \lim_{x\to a} f(x)$ and $\displaystyle \lim_{x\to a}g(x)$ \textbf{both exist}. Then:\\
\begin{enumerate}[1.]\itemsep+15pt
\item $\displaystyle \lim_{x\to a}[f(x)+g(x)]=\lim_{x\to a}f(x)+\lim_{x\to a}g(x)$  \quad ({\em limit of the sum $=$ sum of the limits})
\item $\displaystyle \lim_{x\to a}[f(x)-g(x)]=\lim_{x\to a}f(x)-\lim_{x\to a}g(x)$  
\quad ({\em limit of the difference $=$ difference of the limits})
\item $\displaystyle \lim_{x\to a}cf(x)=c\lim_{x\to a}f(x)$ for any constant $c$  \quad  ({\em constants pull out})
\item $\displaystyle \lim_{x\to a}[f(x)g(x)]=\lim_{x\to a}f(x)\cdot\lim_{x\to a}g(x)$  \quad ({\em limit of the product $=$ product of the limits})
\item $\displaystyle \lim_{x\to a}\frac{f(x)}{g(x)}=\frac{\displaystyle\lim_{x\to a}f(x)}{\displaystyle\lim_{x\to a}g(x)}$
if $\displaystyle \lim_{x\to a}g(x)\neq0$  \quad ({\em limit of the quotient $=$ quotient of the limits})
\item $\displaystyle \lim_{x\to a}\left[f(x)\right]^n=\left[\lim_{x\to a}f(x)\right]^n$
where $n$ is any positive integer   \quad  ({\em this follows from 4.})
\item $\displaystyle \lim_{x\to a} c =c$ for any constant $c$  \quad ({\em the limit of a constant is itself})
\item $\displaystyle \lim_{x\to a} x = a$, and $\displaystyle \lim_{x\to a} x^n = a^n$ where $n$ is any positive integer.
% \item $\displaystyle \lim_{x\to a} \sqrt[n]{x} = \sqrt[n]{a}$ where $n$ is a positive integer.
% (If $n$ is even, we assume that $a>0$.)
% \item $\displaystyle \lim_{x\to a} \sqrt[n]{f(x)} = \sqrt[n]{\lim_{x\to a}f(x)}$ where $n$ is a positive integer.
% (If $n$ is even, we assume that $\displaystyle\lim_{x\to a}f(x)>0$.)
\item $\displaystyle \lim_{x\to a} [f(x)]^{p/r} = [\lim_{x\to a} f(x)]^{p/r}$, where $p$ and $r$ are integers with $r \neq 0$.
\end{enumerate}

$ $

\subsection*{One-sided Limits}
These rules also follow for one-sided limits.

\subsection*{Limit Existence Theorem}
$\displaystyle\lim_{x\to a}f(x)=L$ if and only if
$\displaystyle\lim_{x\to a^-}f(x)=L=\lim_{x\to a^+}f(x)$.
\quad ({\em The left- and right-hand limits must both
  exist and be equal for the general limit to exist.})

\textit{Note:} This, along with one-sided limit laws, is helpful when one (or both) two-sided limits DNE.

\subsection*{Direct Substitution Property}
If $f$ is a polynomial, rational, exponential, or algebraic function, or is one of $\log_b(x)$, $\cos(x)$, or $\sin(x)$,
and $a$ is in the domain of $f$, then $\displaystyle\lim_{x\to a}f(x)=f(a)$. \quad  ({\em Just plug it in!})

\subsection*{Simplification Property}
If $f(x)=g(x)$ when $x\neq a$, then
$\displaystyle \lim_{x\to a}f(x)$ and $\lim_{x\to a}g(x)$ either both exist or both don't exist,
and are equal provided they exist.

\subsection*{The Squeeze Theorem} (Section 2.6)
If $f(x)\leq g(x)\leq h(x)$ when $x$ is near $a$ (except possibly
at $a$) and
    \[
    \lim_{x\to a}f(x)=\lim_{x\to a}h(x)=L,
    \]
then $\displaystyle\lim_{x\to a}g(x)=L$.


\end{document}
