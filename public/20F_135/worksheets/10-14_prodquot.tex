\documentclass[11pt,reqno,final]{amsart}

\pdfcompresslevel=0
\pdfobjcompresslevel=0

\usepackage[dvipsnames]{xcolor}% adds colors
\usepackage{amsmath, amsthm}% {amsfonts, amssymb}

% New Characters
\usepackage[latin1]{inputenc}%
\usepackage[T1]{fontenc}

\usepackage{MnSymbol}
\usepackage[normalem]{ulem}% underlining

\usepackage[theoremfont, largesc]{newpxtext} % different text,math font
\usepackage{newpxmath}

\makeatletter
\DeclareMathRadical{\sqrtsign}{symbols}{112}{largesymbols}{112}
% \let\sqrt=\undefined
% \DeclareRobustCommand\sqrt{\@ifnextchar[\@sqrt{\mathpalette\@x@sqrt}]}
% \def\@x@sqrt#1#2{%
%  \setbox\z@\hbox{$\m@th#1\sqrtsign{\mkern1mu #2}$}
%  \mkern3mu\box\z@}
\makeatother




% Page Typesetting
\usepackage[final]{microtype}
\usepackage{relsize}
\usepackage[margin=1in]{geometry}
\usepackage{framed}
\usepackage{tikz}

\usepackage{setspace}
\onehalfspacing

\usepackage{hyperref}
\hypersetup{
  final,
  pdftitle={Math 135 - Product and Quotient Rules},
  pdfauthor={Bonventre}, 
  linktoc=page,
  pagebackref,
  colorlinks=true,
  citecolor=PineGreen,
  linkcolor=PineGreen,
  linkbordercolor=PineGreen,
}


% Internal References

\usepackage[inline,shortlabels]{enumitem}

% \numberwithin{equation}{section} 
\numberwithin{figure}{section}

\usepackage[nameinlink,capitalise,noabbrev]{cleveref}

\crefname{equation}{}{} % get \cref to behave as \eqref

% \theoremstyle{plain} % bold name, italic text
\newtheorem{theorem}[equation]{Theorem}%
\newtheorem*{theorem*}{Theorem}%
\newtheorem{lemma}[equation]{Lemma}%
\newtheorem{proposition}[equation]{Proposition}%
\newtheorem{corollary}[equation]{Corollary}%
\newtheorem{conjecture}[equation]{Conjecture}%
\newtheorem*{conjecture*}{Conjecture}%
\newtheorem{claim}[equation]{Claim}%
\newtheorem{question}{Question}

\theoremstyle{definition} % bold name, plain text
\newtheorem{definition}[equation]{Definition}%
\newtheorem*{definition*}{Definition}%
\newtheorem{example}[equation]{Example}%
\newtheorem*{example*}{Example}%
\newtheorem{remark}[equation]{Remark}%
\newtheorem{notation}[equation]{Notation}%
\newtheorem{convention}[equation]{Convention}%
\newtheorem{assumption}[equation]{Assumption}%
\newtheorem{exercise}[question]{Exercise}

% ---------- macros
\newcommand{\set}[1]{\left\{#1\right\}}%
\newcommand{\sets}[2]{\left\{ #1 \;|\; #2\right\}}%
\newcommand{\longto}{\longrightarrow}%
\newcommand{\into}{\hookrightarrow}%
\newcommand{\onto}{\twoheadrightarrow}%

\usepackage{harpoon}
\newcommand{\vect}[1]{\text{\overrightharp{\ensuremath{#1}}}}

\newcommand{\del}{\partial}%

\newcommand{\ki}{\chi}
\newcommand{\ksi}{\xi}
\newcommand{\Ksi}{\Xi}

\newcommand{\dlim}{\displaystyle\lim}

% %%%%%%%%%%%%%%%%%%%%%%%%%%%%%%%%%%%%%%%%%%%%%%%%%%%%%%%%%%%%%%%%%%%%%%%%%%%%%%%%%%%%%%%%%%%%%%%%%%%%

\begin{document}


\begin{center}
        \textbf{\Large Math 135, Calculus 1, Fall 2020}\\[10pt]
        {\large 10-14: Product of Quotient Rules}
\end{center}

\thispagestyle{empty}


\renewcommand{\thesection}{\Alph{section}}

% \vspace{-1pt}

Last week, we introduced the \textbf{derivative function} $f'(x)$ of a function $f(x)$, whose evaluation $f'(a)$ at the point $x=a$ is give by:
\begin{itemize}
\item the slope of the tangent line at $x=a$
\item the instantaneous velocity at time $x = a$
\end{itemize}

On Monday, we covered algorithms to help us compute the derivatives of polynomials and exponential functions.
Today, we'll tackle \textbf{products} and \textbf{quotients}

\begin{theorem*}[Product Rule]
        If $f(x)$ and $g(x)$ are differentiable functions, then so is their product $f(x) \cdot g(x)$.
        The derivative of the product is given by
        \begin{framed}
                \[
                        \dfrac{d}{dx}\Big( f(x) \cdot g(x) \Big) = f'(x) \cdot g(x) + f(x) \cdot g'(x).
                \]
        \end{framed}
\end{theorem*}

\begin{example}[Warning]
        The derivative of a product is \textbf{not} equal to the product of the derivatives:
        
        Consider the product $x \cdot x$. Then
        \[
                \dfrac{d}{dx}\big( x \big) \cdot \dfrac{d}{dx}\big( x \big)
                =
                1 \cdot 1
                =
                1
        \]
        but instead
        \[
                \dfrac{d}{dx}\big(x \cdot x \big) =
                \dfrac{d}{dx}\big(x) \cdot x + x \cdot \dfrac{d}{dx}\big( x \big) =
                1 \cdot x + x \cdot 1 = 2x
        \]
        which is what the \textbf{power rule} gives us for the derivative of $x \cdot x = x^2$.
\end{example}

\begin{exercise}
        Use the Product Rule to find $f'(x)$ when $f(x) = (3x^2+1)e^x$. Simlify your answer.
        \vfill
\end{exercise}

\begin{theorem*}[Quotient Rule]
        If $f(x)$ and $g(x)$ are differentiable functions, then so is their quotient $f(x)/g(x)$ whenever $g(x) \neq 0$.
        The derivative of the quotient is given by
        \begin{framed}
                \[
                        \dfrac{d}{dx} \left( \dfrac{f(x)}{g(x)} \right) = \dfrac{f'(x)g(x) - f(x)g'(x)}{(g(x))^2}.
                \]
        \end{framed}
\end{theorem*}

\begin{remark}
        The Quotient Rule can be derived from the product rule:
        Letting $Q(x) = \dfrac{f(x)}{g(x)}$, cross multiply the equation and differentiate both sides with respect to $x$ using the Product Rule.
        Solving for $Q'(x)$ leads to the above formula.
\end{remark}

\newpage

\begin{exercise}
        Use the Quotient Rule to calculate the derivative of $\dfrac{1}{x^4}$ and check your answer against the result obtained from using the Power Rule.
        \vfill
\end{exercise}

\begin{exercise}
        If $g(x) = \dfrac{3x+1}{2x-5}$, find a simplify $g'(x)$.
        \vfill
\end{exercise}

\begin{exercise}
        If $h(x) = \dfrac{e^x}{x^2+1}$, find and simplify $h'(x)$.
        \vfill
\end{exercise}

\begin{exercise}
        Suppose that $f(3) = 5$, $f'(3) = -7$, $g(3) = 2$, and $g'(3) = 1/2$.
        If $H(x) = \dfrac{f(x)}{x g(x)}$, find $H'(3)$.
        \vfill
\end{exercise}

\end{document}
