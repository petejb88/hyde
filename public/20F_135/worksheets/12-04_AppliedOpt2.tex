\documentclass[11pt,reqno,final]{amsart}

\pdfcompresslevel=0
\pdfobjcompresslevel=0

\usepackage[dvipsnames]{xcolor}% adds colors
\usepackage{amsmath, amsthm}% {amsfonts, amssymb}

% New Characters
\usepackage[latin1]{inputenc}%
\usepackage[T1]{fontenc}

\usepackage{MnSymbol}
\usepackage[normalem]{ulem}% underlining

\usepackage[theoremfont, largesc]{newpxtext} % different text,math font
\usepackage{newpxmath}

\makeatletter
\DeclareMathRadical{\sqrtsign}{symbols}{112}{largesymbols}{112}
% \let\sqrt=\undefined
% \DeclareRobustCommand\sqrt{\@ifnextchar[\@sqrt{\mathpalette\@x@sqrt}]}
% \def\@x@sqrt#1#2{%
%  \setbox\z@\hbox{$\m@th#1\sqrtsign{\mkern1mu #2}$}
%  \mkern3mu\box\z@}
\makeatother




% Page Typesetting
\usepackage[final]{microtype}
\usepackage{relsize}
\usepackage[margin=1in]{geometry}
\usepackage{framed}
\usepackage{tikz}

\usepackage{csquotes}

\usepackage{setspace}
\onehalfspacing

\usepackage{hyperref}
\hypersetup{
  final,
  pdftitle={Math 135 - Applied Optimization II},
  pdfauthor={Bonventre}, 
  linktoc=page,
  pagebackref,
  colorlinks=true,
  citecolor=PineGreen,
  linkcolor=PineGreen,
  linkbordercolor=PineGreen,
}


% Internal References

\usepackage[inline,shortlabels]{enumitem}

% \numberwithin{equation}{section} 
\numberwithin{figure}{section}

\usepackage[nameinlink,capitalise,noabbrev]{cleveref}

\crefname{equation}{}{} % get \cref to behave as \eqref

% \theoremstyle{plain} % bold name, italic text
\newtheorem{theorem}[equation]{Theorem}%
\newtheorem*{theorem*}{Theorem}%
\newtheorem{lemma}[equation]{Lemma}%
\newtheorem{proposition}[equation]{Proposition}%
\newtheorem{corollary}[equation]{Corollary}%
\newtheorem{conjecture}[equation]{Conjecture}%
\newtheorem*{conjecture*}{Conjecture}%
\newtheorem{claim}[equation]{Claim}%
\newtheorem{question}{Question}

\theoremstyle{definition} % bold name, plain text
\newtheorem{definition}[equation]{Definition}%
\newtheorem*{definition*}{Definition}%
\newtheorem{example}[equation]{Example}%
\newtheorem*{example*}{Example}%
\newtheorem{remark}[equation]{Remark}%
\newtheorem{notation}[equation]{Notation}%
\newtheorem{convention}[equation]{Convention}%
\newtheorem{assumption}[equation]{Assumption}%
\newtheorem{exercise}[question]{Exercise}

% ---------- macros
\newcommand{\set}[1]{\left\{#1\right\}}%
\newcommand{\sets}[2]{\left\{ #1 \;|\; #2\right\}}%
\newcommand{\longto}{\longrightarrow}%
\newcommand{\into}{\hookrightarrow}%
\newcommand{\onto}{\twoheadrightarrow}%

\usepackage{harpoon}
\newcommand{\vect}[1]{\text{\overrightharp{\ensuremath{#1}}}}

\newcommand{\del}{\partial}%

\newcommand{\ki}{\chi}
\newcommand{\ksi}{\xi}
\newcommand{\Ksi}{\Xi}

\newcommand{\dlim}{\displaystyle\lim}

% %%%%%%%%%%%%%%%%%%%%%%%%%%%%%%%%%%%%%%%%%%%%%%%%%%%%%%%%%%%%%%%%%%%%%%%%%%%%%%%%%%%%%%%%%%%%%%%%%%%%

\begin{document}


\begin{center}
        \textbf{\Large Math 135, Calculus 1, Fall 2020}\\[10pt]
        {\large 12-04: Applied Optimization II}
\end{center}

\thispagestyle{empty}


\renewcommand{\thesection}{\Alph{section}}

Today we will be applying our optimization techniques (EVT and others) to solve word problems.
\begin{itemize}
\item The goal for all of these questions is to translate the word problem into a math question of the form
        ``where is this function of one variable maximized/minimized?''
        This process has many steps: finding the function at hand (and if necessary converting it into a function of one variable), considering the domain, and applying the EVT (or related analysis).
\item Word problems are hard! They are hard for everyone --- students, grad students, professors, economists, politicans, doctors --- everyone.
        It is okay to get discouraged or frustrated. But these are the most important questions: applying calculus techniques and problem solving skills to the ``real world''.
        No one is going to offer you a job because you can take the derivative of a function, but a good problem solver is indispensable. 
\end{itemize}

\begin{exercise}
        League of Legends is a multiplayer online video game. One aspect of the game involves battling other players.
        A player's \textit{Effective Health} when defending against physical damage is given by
        \[
                E = h + \dfrac{ha}{100}
        \]
        where $h$ is the player's \textbf{Health} and $a$ is the player's \textbf{Armor}.
        Players can purchase more Health and Armor with gold coins: Health costs 2.5 coins per unit, and Armor costs 18 coins per unit.
        Suppose a player has 2662 gold coins.
        What is the maximum Effective Health the player can achieve?
        \begin{enumerate}[(a)]
        \item Assume the player will spend all of their coins. Write an equation relating $h$ and $a$.
                \vfill
        \item Use your answer to Part (a) to write an equation for the player's Effective Health in terms of their Health $h$.
                \vfill
                \vfill
                \vfill
        \item What is the domain of this function in the context of this problem?
                \vfill
                \newpage
        \item Maximize this function.
                \vfill
                \vfill
        \item How do you know your value from Part (c) is the absolute maximum (as opposed to a local maximum or an absolute/local minimum)?
                \vfill
        \end{enumerate}
\end{exercise}

\newpage

\begin{exercise}
        A food company wants to design aluminum cans which minimize the amount of metal needed.
        The cans need to hold 12 oz of liquid, or approximately 21.7 in$^3$.
        \begin{enumerate}
        \item Write an equation for the surface area $A$ of a cylindrical can in terms of the radius $r$ and the height $h$.
                \vfill
        \item Use the constraint that the cans need to hold 12 oz of liquid to write an equation relating $r$ and $h$.
                \vfill
        \item Combine Parts (a) and (b) to write an equation for the surface area $A$ as a function of the radius $r$.
                \vfill
                \vfill
        \item What is the domain of this function in this context?
                \vfill
                \newpage
        \item Minimize this function. Justify your answer.
                \vfill
                \vfill
                \vfill
        \item What does your answer to Part (c) mean in the context of this problem?
                \vfill
        \item What are the dimensions of the can the company should make?
                \vfill
        \end{enumerate}
\end{exercise}

\end{document}
