\documentclass[11pt,reqno,final]{amsart}

\pdfcompresslevel=0
\pdfobjcompresslevel=0

\usepackage[dvipsnames]{xcolor}% adds colors
\usepackage{amsmath, amsthm}% {amsfonts, amssymb}

% New Characters
\usepackage[latin1]{inputenc}%
\usepackage[T1]{fontenc}

\usepackage{MnSymbol}
\usepackage[normalem]{ulem}% underlining

\usepackage[theoremfont, largesc]{newpxtext} % different text,math font
\usepackage{newpxmath}

\makeatletter
\DeclareMathRadical{\sqrtsign}{symbols}{112}{largesymbols}{112}
% \let\sqrt=\undefined
% \DeclareRobustCommand\sqrt{\@ifnextchar[\@sqrt{\mathpalette\@x@sqrt}]}
% \def\@x@sqrt#1#2{%
%  \setbox\z@\hbox{$\m@th#1\sqrtsign{\mkern1mu #2}$}
%  \mkern3mu\box\z@}
\makeatother




% Page Typesetting
\usepackage[final]{microtype}
\usepackage{relsize}
\usepackage[margin=1in]{geometry}
\usepackage{framed}
\usepackage{tikz}
\usepackage{setspace}

\usepackage{hyperref}
\hypersetup{
  final,
  pdftitle={Math 135 - Algebraic Limits},
  pdfauthor={Bonventre}, 
  linktoc=page,
  pagebackref,
  colorlinks=true,
  citecolor=PineGreen,
  linkcolor=PineGreen,
  linkbordercolor=PineGreen,
}


% Internal References

\usepackage[inline,shortlabels]{enumitem}

% \numberwithin{equation}{section} 
\numberwithin{figure}{section}

\usepackage[nameinlink,capitalise,noabbrev]{cleveref}

\crefname{equation}{}{} % get \cref to behave as \eqref

% \theoremstyle{plain} % bold name, italic text
\newtheorem{theorem}[equation]{Theorem}%
\newtheorem*{theorem*}{Theorem}%
\newtheorem{lemma}[equation]{Lemma}%
\newtheorem{proposition}[equation]{Proposition}%
\newtheorem{corollary}[equation]{Corollary}%
\newtheorem{conjecture}[equation]{Conjecture}%
\newtheorem*{conjecture*}{Conjecture}%
\newtheorem{claim}[equation]{Claim}%
\newtheorem{question}{Question}

\theoremstyle{definition} % bold name, plain text
\newtheorem{definition}[equation]{Definition}%
\newtheorem*{definition*}{Definition}%
\newtheorem{example}[equation]{Example}%
\newtheorem*{example*}{Example}%
\newtheorem{remark}[equation]{Remark}%
\newtheorem{notation}[equation]{Notation}%
\newtheorem{convention}[equation]{Convention}%
\newtheorem{assumption}[equation]{Assumption}%
\newtheorem{exercise}[question]{Exercise}

% ---------- macros
\newcommand{\set}[1]{\left\{#1\right\}}%
\newcommand{\sets}[2]{\left\{ #1 \;|\; #2\right\}}%
\newcommand{\longto}{\longrightarrow}%
\newcommand{\into}{\hookrightarrow}%
\newcommand{\onto}{\twoheadrightarrow}%

\usepackage{harpoon}
\newcommand{\vect}[1]{\text{\overrightharp{\ensuremath{#1}}}}

\newcommand{\del}{\partial}%

\newcommand{\ki}{\chi}
\newcommand{\ksi}{\xi}
\newcommand{\Ksi}{\Xi}

\newcommand{\dlim}{\displaystyle\lim}

% %%%%%%%%%%%%%%%%%%%%%%%%%%%%%%%%%%%%%%%%%%%%%%%%%%%%%%%%%%%%%%%%%%%%%%%%%%%%%%%%%%%%%%%%%%%%%%%%%%%%

\begin{document}
\onehalfspacing

\begin{center}
        \textbf{\Large Math 135, Calculus 1, Fall 2020}\\[10pt]
        {\large 09-28: Evaluating Limits Algebraically}
\end{center}

\thispagestyle{empty}

\renewcommand{\thesection}{\Alph{section}}

\vspace{-1pt}
\section{Indeterminant forms}

We have looked at several ways to evaluate the limit $\dlim_{x \to a}f(x)$:
\begin{enumerate}[(a)]
\item If we know the function $f(x)$ is continuous at $x = a$, then the limit is simply $f(a)$.
\item If we have the \textit{graph} of the function $f$, we can visually determine the limit.
\item We can perform numerical calculations (i.e. plug in values really, really close to $a$) and make a guess about the limit based on this information.
\item We can use algebra to make the calculation easier.
\end{enumerate}

\begin{example}
        Recall the limit $\dlim_{x \to 3^+}\dfrac{1}{x-3}$ considered on 09-18.
        Plugging in $x=3$, we get $1/0$, which does not exist.
        However, it could be $+\infty$ or $-\infty$. Let's see:
        \begin{enumerate}[(a)]\setcounter{enumi}{1}
        \item If we had the graph, we would see that the function values blow up as $x \to 3^+$ (``$x$ approaches 3 from the right'').
        \item Testing $x = 3.0001$, we get $f(3.0001) = 1/(0.0001) = 10000$ which just gets bigger if we add more zeros. Hence it makes sense to conclude that the limit is $+\infty$.
        \item Algebraically, we have that as $x \to 3^+$, $(x-3) \to 0^+$, so we have that the limit can be expressed as
                $1/0^+ = +\infty$.
        \end{enumerate}
\end{example}

\begin{example}
        Consider the limit $\dlim_{x \to -\infty}\dfrac{6x^4-5x^2+1}{3x^3-15}$.
        ``Evaluating'', we would get
        \[                
                \dfrac{6(-\infty)^4 - (-\infty)^2+1}{3(-\infty)^3 - 15}\ \mbox{ `` = '' }\  \dfrac{\infty - \infty}{-\infty}.
        \]
        This expression contains two \textbf{indeterminant forms}, and thus gives us \textbf{no information}.
\end{example}

\begin{framed}
        The key \textbf{indeterminante forms} are $\dfrac{0}{0}$, $\dfrac{\infty}{\infty}$, $\infty \cdot 0$, and $\infty - \infty$.\\\

        A limit that takes one of these forms can by \textit{anything} (any value at all: a real number $L$, $+\infty$, $-\infty$, or DNE), and thus we can make \textbf{no conclusions whatsoever} about the limit based on this evaluation.
        Instead, \textit{further algebraic work must be done} to find the actual value of the limit.        
\end{framed}


\begin{example*}[Example 2, Continued: Technique for Infinite Limits]
        To get rid of these indeterminant forms, we first multiply this rational function by
        $1 = \frac{1}{x^3}/\frac{1}{x^3}$
        (where 3 is the highest power of $x$ in the denominator) to get
        \[
                \dfrac{6x^4-5x^2+1}{3x^3-15}
                = \dfrac{\frac{6x^4}{x^3} - \frac{5x^2}{x^3} + \frac{1}{x^3}}{\frac{3x^3}{x^3} -\frac{15}{x^3}}
                = \dfrac{6x - \frac{5}{x} + \frac{1}{x^3}}{3 - \frac{15}{x^3}}.
        \]
        We can now evaluate this limit, and get
        \[
                \dlim_{x \to -\infty}\dfrac{6x^4-5x^2+1}{3x^3-15}
                = \dlim_{x \to -\infty}\dfrac{6x - \frac{5}{x} + \frac{1}{x^3}}{3 - \frac{15}{x^3}}
                = \dlim_{x \to -\infty}\dfrac{-6\infty - 0}{3-0}
                = -\infty.
        \]
\end{example*}

\newpage

\begin{exercise}
        Compute $\dlim_{x \to -\infty}\dfrac{5x^3-2x^2+3}{-2x^2+1}$.
        \vfill
\end{exercise}

\section{Changing one value}
Recall the following key observation:
\begin{framed}
        The value of $f(a)$ (or even if it exists) has \textbf{no effect} on the value of $\dlim_{x \to a}f(x)$.
\end{framed}
This means that if we change the function only at $x=a$, the limit is unchanged.

\begin{exercise}[General technique for $\frac{0}{0}$]
        Find the value of the limit by first canceling a common factor from the numerator and the denominator.
        What value of the function have we changed?
        \[
                \dlim_{x \to 2}\dfrac{x^2-4}{x^2+4x-12}
        \]
        \vfill
\end{exercise}

\begin{exercise}
        If $f(x) = 5x^2-3x$, find the value of $\dlim_{h \to 0}\dfrac{f(3+h)-f(3)}{h}$.
        \vfill          
\end{exercise}

\begin{exercise}[Technique for functions with square roots]
        Find the value of $\dlim_{x \to 9}\dfrac{\sqrt x - 3}{9-x}$.
        (\textit{Hint: multiply top and bottom by the conjugate $\sqrt x +3$ of the numerator.)}
        \vfill
\end{exercise}

\newpage

\begin{exercise}
        Evaluate $\dlim_{\theta \to \pi/2} \dfrac{\tan \theta}{\sec \theta}$.
        \vfill
\end{exercise}

\section{Other Techinques}
\begin{exercise}[Technique for functions with absolute values]
        Find the value of each one-sided limits.
        \textit{Hint: use the definition of the absolute value function, and the property $|ab| = |a||b|$.}\\
        \begin{enumerate*}[(i)]
        \item $\dlim_{x \to 3^-} \dfrac{|4x-12|}{x-3}$
                \qquad \qquad \qquad $ $ \qquad \qquad \qquad
        \item $\dlim_{x \to 3^+} \dfrac{|4x-12|}{x-3}$
        \end{enumerate*}
        \vfill
\end{exercise}

\begin{exercise}
        Find the value of $\dlim_{t \to 1}\dfrac{6}{t^2 - 1} - \dfrac{3}{t-1}$.
        \textit{Hint: add the fractions and simplify.}
        \vfill
\end{exercise}

\end{document}
