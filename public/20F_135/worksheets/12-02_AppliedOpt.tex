\documentclass[11pt,reqno,final]{amsart}

\pdfcompresslevel=0
\pdfobjcompresslevel=0

\usepackage[dvipsnames]{xcolor}% adds colors
\usepackage{amsmath, amsthm}% {amsfonts, amssymb}

% New Characters
\usepackage[latin1]{inputenc}%
\usepackage[T1]{fontenc}

\usepackage{MnSymbol}
\usepackage[normalem]{ulem}% underlining

\usepackage[theoremfont, largesc]{newpxtext} % different text,math font
\usepackage{newpxmath}

\makeatletter
\DeclareMathRadical{\sqrtsign}{symbols}{112}{largesymbols}{112}
% \let\sqrt=\undefined
% \DeclareRobustCommand\sqrt{\@ifnextchar[\@sqrt{\mathpalette\@x@sqrt}]}
% \def\@x@sqrt#1#2{%
%  \setbox\z@\hbox{$\m@th#1\sqrtsign{\mkern1mu #2}$}
%  \mkern3mu\box\z@}
\makeatother




% Page Typesetting
\usepackage[final]{microtype}
\usepackage{relsize}
\usepackage[margin=1in]{geometry}
\usepackage{framed}
\usepackage{tikz}

\usepackage{csquotes}

\usepackage{setspace}
\onehalfspacing

\usepackage{hyperref}
\hypersetup{
  final,
  pdftitle={Math 135 - Applied Optimization},
  pdfauthor={Bonventre}, 
  linktoc=page,
  pagebackref,
  colorlinks=true,
  citecolor=PineGreen,
  linkcolor=PineGreen,
  linkbordercolor=PineGreen,
}


% Internal References

\usepackage[inline,shortlabels]{enumitem}

% \numberwithin{equation}{section} 
\numberwithin{figure}{section}

\usepackage[nameinlink,capitalise,noabbrev]{cleveref}

\crefname{equation}{}{} % get \cref to behave as \eqref

% \theoremstyle{plain} % bold name, italic text
\newtheorem{theorem}[equation]{Theorem}%
\newtheorem*{theorem*}{Theorem}%
\newtheorem{lemma}[equation]{Lemma}%
\newtheorem{proposition}[equation]{Proposition}%
\newtheorem{corollary}[equation]{Corollary}%
\newtheorem{conjecture}[equation]{Conjecture}%
\newtheorem*{conjecture*}{Conjecture}%
\newtheorem{claim}[equation]{Claim}%
\newtheorem{question}{Question}

\theoremstyle{definition} % bold name, plain text
\newtheorem{definition}[equation]{Definition}%
\newtheorem*{definition*}{Definition}%
\newtheorem{example}[equation]{Example}%
\newtheorem*{example*}{Example}%
\newtheorem{remark}[equation]{Remark}%
\newtheorem{notation}[equation]{Notation}%
\newtheorem{convention}[equation]{Convention}%
\newtheorem{assumption}[equation]{Assumption}%
\newtheorem{exercise}[question]{Exercise}

% ---------- macros
\newcommand{\set}[1]{\left\{#1\right\}}%
\newcommand{\sets}[2]{\left\{ #1 \;|\; #2\right\}}%
\newcommand{\longto}{\longrightarrow}%
\newcommand{\into}{\hookrightarrow}%
\newcommand{\onto}{\twoheadrightarrow}%

\usepackage{harpoon}
\newcommand{\vect}[1]{\text{\overrightharp{\ensuremath{#1}}}}

\newcommand{\del}{\partial}%

\newcommand{\ki}{\chi}
\newcommand{\ksi}{\xi}
\newcommand{\Ksi}{\Xi}

\newcommand{\dlim}{\displaystyle\lim}

% %%%%%%%%%%%%%%%%%%%%%%%%%%%%%%%%%%%%%%%%%%%%%%%%%%%%%%%%%%%%%%%%%%%%%%%%%%%%%%%%%%%%%%%%%%%%%%%%%%%%

\begin{document}


\begin{center}
        \textbf{\Large Math 135, Calculus 1, Fall 2020}\\[10pt]
        {\large 12-02: Applied Optimization}
\end{center}

\thispagestyle{empty}


\renewcommand{\thesection}{\Alph{section}}

Today we will be applying our optimization techniques (EVT and others) to solve word problems.
\begin{itemize}
\item The goal for all of these questions is to translate the word problem into a math question of the form
        ``where is this function of one variable maximized/minimized?''
        This process has many steps: finding the function at hand (and if necessary converting it into a function of one variable), considering the domain, and applying the EVT (or related analysis).
\item Word problems are hard! They are hard for everyone --- students, grad students, professors, economists, politicans, doctors --- everyone.
        It is okay to get discouraged or frustrated. But these are the most important questions: applying calculus techniques and problem solving skills to the ``real world''.
        No one is going to offer you a job because you can take the derivative of a function, but a good problem solver is indispensable. 
\end{itemize}


\begin{exercise}
        A farmer has 2400 feet of fencing and wants to use it to fence off a rectangular field. What are the dimensions of the field that has the largest area, and what is that largest area?\\
        \textit{Remember, the goal is to model this situation with a fucntion of one variable, and then optimize this function.}
        \begin{enumerate}[(a)]
        \item Draw a picture of several possible fields. Label the pictures by assigning variables to any quantities that change. List any other variablese that might be important.
                \vfill
                \vfill
                \vfill
        \item Which quantity from Part (a) is the one that we want to maximize?
                \vfill
        \item Use basic geometry to write a formula for the variable you named in Part (b), in terms of other variables you identified from Part (a).
                You may end up with a function that has two input variables --- that's okay! We will fix that in the next step.
                \vfill
                \vfill
                \newpage
        \item Turn the constraint that we have only 2400 feet of fencing into an equation involving your variables from Part (a).
                Then use this equation to eliminate one of the variables from your function in Part (c).
                \textit{Your result should be a function of one variable, and this is the function to maximize.}
                \vfill
                \vfill
        \item What is the domain of your function, \textbf{in the context of this problem?}
                (You can allow for ``silly'' rectangles with no area.)
                \vfill
        \item Use one of the procedures you know to find the absolute max value on the domain.
                \textit{(Does the EVT apply? If not, what is the concavity of this function on the domain?)}
                \vfill
                \vfill
        \item Answer the questions asked: what are the dimensions of the field that has the largest area, and what is that largest area?
                \vfill
        \end{enumerate}
\end{exercise}

\newpage

\begin{exercise}
        A farmer has 2400 feet of fencing, and this time wants to fence off a rectangular field that borders a straight river. The farmer needs no fence along the river.
        What are the dimensions of the field that has the largest area, and what is that largest area?
        (This problem is similar to Exercise 1; use the same sequence of steps in your solution.)
        Explain why your answer is different from Exercise 1.
\end{exercise}

\newpage

\begin{exercise}
        A jeweler wants to make a square-bottomed box with no top that has a volume of 500 cm$^3$.
        What are the dimensions that minimize the surface area of the box?
        \vfill
\end{exercise}

\begin{exercise}
        The same jeweler wants another square-bottomed box with no top with the same volume of 500cm$^3$.
        But this time, the material for the bottom costs \$2 per cm$^2$ while the sides cost \$1 per cm$^2$.
        In this case, what dimensions give the box with the lowest cost?
        \vfill
\end{exercise}

\end{document}
