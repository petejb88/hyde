\documentclass[11pt,reqno,final]{amsart}

\pdfcompresslevel=0
\pdfobjcompresslevel=0

\usepackage[dvipsnames]{xcolor}% adds colors
\usepackage{amsmath, amsthm}% {amsfonts, amssymb}

% New Characters
\usepackage[latin1]{inputenc}%
\usepackage[T1]{fontenc}

\usepackage{MnSymbol}
\usepackage[normalem]{ulem}% underlining

\usepackage[theoremfont, largesc]{newpxtext} % different text,math font
\usepackage{newpxmath}

\makeatletter
\DeclareMathRadical{\sqrtsign}{symbols}{112}{largesymbols}{112}
% \let\sqrt=\undefined
% \DeclareRobustCommand\sqrt{\@ifnextchar[\@sqrt{\mathpalette\@x@sqrt}]}
% \def\@x@sqrt#1#2{%
%  \setbox\z@\hbox{$\m@th#1\sqrtsign{\mkern1mu #2}$}
%  \mkern3mu\box\z@}
\makeatother




% Page Typesetting
\usepackage[final]{microtype}
\usepackage{relsize}
\usepackage[margin=1in]{geometry}
\usepackage{framed}
\usepackage{tikz}

\usepackage{csquotes}

\usepackage{setspace}
\onehalfspacing

\usepackage{hyperref}
\hypersetup{
  final,
  pdftitle={Math 135 - Logarithmic Differentiation},
  pdfauthor={Bonventre}, 
  linktoc=page,
  pagebackref,
  colorlinks=true,
  citecolor=PineGreen,
  linkcolor=PineGreen,
  linkbordercolor=PineGreen,
}


% Internal References

\usepackage[inline,shortlabels]{enumitem}

% \numberwithin{equation}{section} 
\numberwithin{figure}{section}

\usepackage[nameinlink,capitalise,noabbrev]{cleveref}

\crefname{equation}{}{} % get \cref to behave as \eqref

% \theoremstyle{plain} % bold name, italic text
\newtheorem{theorem}[equation]{Theorem}%
\newtheorem*{theorem*}{Theorem}%
\newtheorem{lemma}[equation]{Lemma}%
\newtheorem{proposition}[equation]{Proposition}%
\newtheorem{corollary}[equation]{Corollary}%
\newtheorem{conjecture}[equation]{Conjecture}%
\newtheorem*{conjecture*}{Conjecture}%
\newtheorem{claim}[equation]{Claim}%
\newtheorem{question}{Question}

\theoremstyle{definition} % bold name, plain text
\newtheorem{definition}[equation]{Definition}%
\newtheorem*{definition*}{Definition}%
\newtheorem{example}[equation]{Example}%
\newtheorem*{example*}{Example}%
\newtheorem{remark}[equation]{Remark}%
\newtheorem{notation}[equation]{Notation}%
\newtheorem{convention}[equation]{Convention}%
\newtheorem{assumption}[equation]{Assumption}%
\newtheorem{exercise}[question]{Exercise}

% ---------- macros
\newcommand{\set}[1]{\left\{#1\right\}}%
\newcommand{\sets}[2]{\left\{ #1 \;|\; #2\right\}}%
\newcommand{\longto}{\longrightarrow}%
\newcommand{\into}{\hookrightarrow}%
\newcommand{\onto}{\twoheadrightarrow}%

\usepackage{harpoon}
\newcommand{\vect}[1]{\text{\overrightharp{\ensuremath{#1}}}}

\newcommand{\del}{\partial}%

\newcommand{\ki}{\chi}
\newcommand{\ksi}{\xi}
\newcommand{\Ksi}{\Xi}

\newcommand{\dlim}{\displaystyle\lim}

% %%%%%%%%%%%%%%%%%%%%%%%%%%%%%%%%%%%%%%%%%%%%%%%%%%%%%%%%%%%%%%%%%%%%%%%%%%%%%%%%%%%%%%%%%%%%%%%%%%%%

\begin{document}


\begin{center}
        \textbf{\Large Math 135, Calculus 1, Fall 2020}\\[10pt]
        {\large 11-02: Logarithmic Differentiation (Section 3.8) and Rates of Change (Section 3.4)}
\end{center}

\thispagestyle{empty}


\renewcommand{\thesection}{\Alph{section}}

% \vspace{-1pt}

The \textbf{derivative} $f'(x)$ of a function $y=f(x)$ gives:
\begin{itemize}
\item the slope of the tangent line
\item the instantaneous rate of change of $y$ with respect to $x$
\end{itemize}


\section{Logarithmic Differentiation}

\begin{exercise}
        Consider the function $f(x) = (3x-2)^x$.
        \begin{enumerate}[(a)]
        \item Take the natural log of both sides of this equation, and simplify the right-hand-side by using log rules:
                \[
                        \ln(a \cdot b) = \ln(a) + \ln(b),
                        \qquad
                        \ln\left( \frac{a}{b} \right) = \ln(a) - \ln(b),
                        \qquad
                        \ln\left(a^b\right) = b \cdot \ln(a).
                \]
                \vfill
        \item Use implicit differentiation to compute $f'(x)$ in terms of $x$ and $f(x)$.
                \vfill
        \item Replace $f(x)$ with the original expression to find $f'(x)$ just in terms of $x$.
                \vfill
        \end{enumerate}
\end{exercise}

\newpage

\section{Rates of Change}

\begin{framed}
        For any function $y = f(x)$, the derviative $\dfrac{dy}{dx}$ measures the
        \textbf{instantaneous rate of change} of $y$ with respect to $x$.
\end{framed}

\begin{example}
        If $T(t)$ measures the temperature $T$ (in degrees Celsius) of an object as a function of time $t$ (in seconds),
        then $\dfrac{dT}{dt}$ measures the rate the temperature of the object is changing.
        The units of $\dfrac{dT}{dt}$ are $^\circ C/sec$.
        If $\dfrac{dT}{dt} > 0$, then the object is warming;
        if $\dfrac{dT}{dt} < 0$, then the object is cooling.
\end{example}

\subsection*{Application to Economics}
Let $C(x)$ be the cost of producing a quantity $x$ of some item,
e.g. $C(25) = \$ 3000$ means it costs \$3000 to produce 25 items.
The derivative $C'(x)$ is called the \textbf{marginal cost}, and gives an approximation to the cost of producing the $(x+1)$-st item.
Similarly:
\begin{itemize}
\item if $P(x)$ is the profit made from selling $x$ items, then $P'(x)$ is called the \textbf{marginal profit}, and
\item if $R(x)$ is the revenue made from selling $x$ items, then $R'(x)$ is called the \textbf{marginal revenue}.
\end{itemize}

\begin{exercise}
        Suppose $C(x) = 8000-10x+x^2 + 0.01x^3$ represents the cost of producing $x$ computers.
        \begin{enumerate}[(a)]
        \item Find the marginal cost function.
                \vfill
        \item Find $C'(10)$ and explain its meaing. What are the units of $C'(10)$?
                \vfill
        \item Find the actual cost of producing the 11th computer. Compare your answer with $C'(10)$.
                \vfill
        \end{enumerate}
\end{exercise}

\newpage

\subsection*{Application to Physics}

If $s(t)$ is the position of a moving object as a function of time $t$, then
$s'(t) = v(t)$ is the \textbf{instantaneous velocity}, and
$s''(t) = v'(t) = a(t)$ is the \textbf{instantaneous acceleration}.
The speed of the object is defined to be $|s'(t)| = |v(t)|$, which is always positive.

\begin{exercise}
        Suppose a particle moves according to the equation $s(t) = t^3 - 12t^2 + 36t$ for $t \geq 0$,
        where $s$, the position, is measured in meters, and $t$, the time, is measured in seconds.\\
        \textit{Think of the particle moving along a number line, with $s$ indicating the position on the line.}
        \begin{enumerate}[(a)]
        \item Compute the velocity and acceleration of the particle at time $t$.
                \vfill
        \item When is the particle at rest?
                \vfill
        \item What is the particle moving to the right? to the left?
                \vfill
        \item Find the total distance traveled by the particle in the first 6 seconds.
                \vfill
        \end{enumerate}
        
\end{exercise}


\end{document}
