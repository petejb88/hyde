\documentclass[11pt,reqno,final]{amsart}

\pdfcompresslevel=0
\pdfobjcompresslevel=0

\usepackage[dvipsnames]{xcolor}% adds colors
\usepackage{amsmath, amsthm}% {amsfonts, amssymb}

% New Characters
\usepackage[latin1]{inputenc}%
\usepackage[T1]{fontenc}

\usepackage{MnSymbol}
\usepackage[normalem]{ulem}% underlining

\usepackage[theoremfont, largesc]{newpxtext} % different text,math font
\usepackage{newpxmath}

\makeatletter
\DeclareMathRadical{\sqrtsign}{symbols}{112}{largesymbols}{112}
% \let\sqrt=\undefined
% \DeclareRobustCommand\sqrt{\@ifnextchar[\@sqrt{\mathpalette\@x@sqrt}]}
% \def\@x@sqrt#1#2{%
%  \setbox\z@\hbox{$\m@th#1\sqrtsign{\mkern1mu #2}$}
%  \mkern3mu\box\z@}
\makeatother




% Page Typesetting
\usepackage[final]{microtype}
\usepackage{relsize}
\usepackage[margin=1in,bottom=0in]{geometry}
\usepackage{framed}
\usepackage{tikz}
\usepackage{setspace}

\usepackage{hyperref}
\hypersetup{
  final,
  pdftitle={Math 135 - Project 3},
  pdfauthor={Bonventre}, 
  linktoc=page,
  pagebackref,
  colorlinks=true,
  citecolor=PineGreen,
  linkcolor=PineGreen,
  linkbordercolor=PineGreen,
}


% Internal References

\usepackage[inline,shortlabels]{enumitem}

\numberwithin{equation}{section} 
\numberwithin{figure}{section}

\usepackage[nameinlink,capitalise,noabbrev]{cleveref}

\crefname{equation}{}{} % get \cref to behave as \eqref

% \theoremstyle{plain} % bold name, italic text
\newtheorem{theorem}[equation]{Theorem}%
\newtheorem*{theorem*}{Theorem}%
\newtheorem{lemma}[equation]{Lemma}%
\newtheorem{proposition}[equation]{Proposition}%
\newtheorem{corollary}[equation]{Corollary}%
\newtheorem{conjecture}[equation]{Conjecture}%
\newtheorem*{conjecture*}{Conjecture}%
\newtheorem{claim}[equation]{Claim}%


\theoremstyle{definition} % bold name, plain text
\newtheorem{definition}[equation]{Definition}%
\newtheorem*{definition*}{Definition}%
\newtheorem{example}[equation]{Example}%
\newtheorem*{example*}{Example}%
\newtheorem{remark}[equation]{Remark}%
\newtheorem{notation}[equation]{Notation}%
\newtheorem{convention}[equation]{Convention}%
\newtheorem{assumption}[equation]{Assumption}%
\newtheorem{exercise}{Exercise}
\newtheorem{problem}{Problem}

% ---------- macros
\newcommand{\set}[1]{\left\{#1\right\}}%
\newcommand{\sets}[2]{\left\{ #1 \;|\; #2\right\}}%
\newcommand{\longto}{\longrightarrow}%
\newcommand{\into}{\hookrightarrow}%
\newcommand{\onto}{\twoheadrightarrow}%

\usepackage{harpoon}
\newcommand{\vect}[1]{\text{\overrightharp{\ensuremath{#1}}}}

\newcommand{\del}{\partial}%

\newcommand{\ki}{\chi}
\newcommand{\ksi}{\xi}
\newcommand{\Ksi}{\Xi}

\newcommand{\dlim}{\displaystyle\lim}

\DeclareMathOperator{\sech}{sech}

% %%%%%%%%%%%%%%%%%%%%%%%%%%%%%%%%%%%%%%%%%%%%%%%%%%%%%%%%%%%%%%%%%%%%%%%%%%%%%%%%%%%%%%%%%%%%%%%%%%%%

\begin{document}
\onehalfspacing

\begin{center}
        \textbf{\Large Math 135, Calculus 1, Fall 2020}\\[10pt]
        {\large Project 3: Approximating Functions}\\
        Due: \textbf{Friday, Dec 11} by 11:59pm.
\end{center}

\thispagestyle{empty}

\renewcommand{\thesection}{\Alph{section}}


\section*{The Project}

In this project, you will investigate the function $f(x) = e^x \sin(x)$ on the domain $[-2\pi, 2\pi]$, as well as several approximations of this function.

\subsection*{Components}

For this project, you will need to submit mathematical solutions to various questions, as well as a single graph.
You must submit your project as a single file or Google Document.

For these questions, you must \textbf{show all work}, and use the Calculus techniques discussed in class. Answers not supported in this way will \textbf{not} be accepted.

$ $

\section{The function $f(x) = e^x \sin(x)$ [40 points]}


\begin{problem}
        Find all critical points of $f(x)$ on $[-2\pi, 2\pi]$. How many are there?
\end{problem}

\begin{problem}
        Create a sign chart for the first derivative of $f(x)$ on $[-2\pi, 2\pi]$. Show all work.
\end{problem}

\begin{problem}
        Create a sign chart for the second derivative of $f(x)$ on $[-2\pi, 2\pi]$. Show all work. Find the inflection points on $[-2\pi, 2\pi]$. How many are there?
\end{problem}

\begin{problem}
        Classify the critical points of $f(x)$ on $[-2\pi, 2\pi]$ using both the First and Second Derivative Tests.
\end{problem}

\begin{problem}
        Find the absolute maximum and absolute minimum values of $f(x)$ on $[-2\pi, 2\pi]$ using the Extreme Value Theorem.
\end{problem}

$ $

\section{Approximations of $f(x) = e^x \sin(x)$ [45 points]}

\subsection{Linearization}
The \textit{linearization} of a function $f(x)$ at $x = a$ is a linear function
\[
        L(x) = A + Bx
\]
such that
\[
        L(a) = f(a), \qquad L'(a) = f'(a).
\]
$L(x)$ gives the best approximation of $f(x)$ near $x = a$ as a line.

\begin{problem}
        Find $L(x)$ for $f(x) = e^x \sin(x)$ at $x = 0$. Show all work.\\
        \textit{(Hint: We know that $L(x)$ must be of the form $A+ Bx$. Can we use the properties of $L(x)$ listed above to find the values of $A$ and $B$?)}
\end{problem}

\begin{problem}
        How many critical points does $L(x)$ have in $[-2\pi, 2\pi]$?
        How many inflection points does $L(x)$ have in $[-2\pi, 2\pi]$?        
\end{problem}

\begin{problem}
        Compute the absolute max and min values of $L(x)$ on $[-2\pi, 2\pi]$ using the EVT.
\end{problem}

\newpage

\subsection{Quadratic approximation}
The \textit{quadratic approximation} of a function $f(x)$ at $x = a$ is a quadratic function
\[
        Q(x) = A + Bx + Cx^2
\]
such that
\[
        Q(a) = f(a),
        \qquad
        Q'(a) = f'(a),
        \qquad
        Q''(a) = f''(a).
\]
$Q(x)$ gives the best approximation of $f(x)$ near $x = a$ as a parabola.

\begin{problem}
        Find $Q(x)$ for $f(x) = e^x \sin(x)$ at $x = 0$. Show all work. \\
        \textit{(Hint: Can we use the properties of $Q(x)$ listed above to find the necessary values of $A$, $B$, and $C$?)}
\end{problem}

\begin{problem}
        How many critical points does $Q(x)$ have in $[-2\pi, 2\pi]$?
        How many inflection points does $Q(x)$ have in $[-2\pi, 2\pi]$?        
\end{problem}

\begin{problem}
         Compute the absolute max and min values of $Q(x)$ on $[-2\pi, 2\pi]$ using the EVT.
\end{problem}

$ $ 

\subsection{Cubic approximation}
The \textit{cubic approximation} of a function $f(x)$ at $x = a$ is a cubic function
\[
        P(x) = A + Bx + Cx^2 + Dx^3
\]
such that
\[
        P(a) = f(a), \qquad
        P'(a) = f'(a), \qquad
        P''(a) = f''(a), \qquad
        P'''(a) = f'''(a).
\]
$P(x)$ is the best approximation of $f(x)$ near $x = a$ as a cubic function.

\begin{problem}
        Find $P(x)$ for $f(x) = e^x \sin(x)$ at $x = 0$. Show all work.\\
        \textit{(Hint: Can we use the properties of $O(x)$ listed above to find the necessary values of $A$, $B$, $C$, and $D$?)}
\end{problem}

\begin{problem}
        How many critical points does $P(x)$ have in $[-2\pi, 2\pi]$?
        How many inflection points does $P(x)$ have in $[-2\pi, 2\pi]$?        
\end{problem}

\begin{problem}
         Compute the absolute max and min values of $P(x)$ on $[-2\pi, 2\pi]$ using the EVT.
\end{problem}

$ $

\subsection{Graph}

\begin{problem}
        Using Desmos, plot $f(x)$, $L(x)$, $Q(x)$, and $P(x)$ on the same set of axes. Make sure the domain is at least $[-2\pi, 2\pi]$.
\end{problem}

$ $

\section{General Approximations [15 points]} 

Suppose $f(x)$ is a function such that $f(x)$, $f'(x)$, $f''(x)$, and $f'''(x)$ are all continuous,
and let $x = a$ be some point in the domain of $f$.

The following formula should be in terms of $a$, $f(a)$, $f'(a)$, $f''(a)$, and $f'''(a)$.

\begin{problem}
        Find a general rule for $L(x)$.
\end{problem}

\begin{problem}
        Find a general rule for $Q(x)$.
\end{problem}

\begin{problem}
        Find a general rule for $P(x)$.
\end{problem}

\end{document}
