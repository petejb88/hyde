\documentclass[11pt,reqno,final]{amsart}

\pdfcompresslevel=0
\pdfobjcompresslevel=0

\usepackage[dvipsnames]{xcolor}% adds colors
\usepackage{amsmath, amsthm}% {amsfonts, amssymb}

% New Characters
\usepackage[latin1]{inputenc}%
\usepackage[T1]{fontenc}

\usepackage{MnSymbol}
\usepackage[normalem]{ulem}% underlining

\usepackage[theoremfont, largesc]{newpxtext} % different text,math font
\usepackage{newpxmath}

\makeatletter
\DeclareMathRadical{\sqrtsign}{symbols}{112}{largesymbols}{112}
\let\sqrt=\undefined
\DeclareRobustCommand\sqrt{\@ifnextchar[\@sqrt{\mathpalette\@x@sqrt}}
\def\@x@sqrt#1#2{%
 \setbox\z@\hbox{$\m@th#1\sqrtsign{\mkern1mu #2}$}
 \mkern3mu\box\z@}
\makeatother




% Page Typesetting
\usepackage[final]{microtype}
\usepackage{relsize}
\usepackage[margin=1in]{geometry}
\usepackage{framed}
\usepackage{tikz}

\usepackage{hyperref}
\hypersetup{
  final,
  pdftitle={Math 135 - Quiz 10-07},
  pdfauthor={Bonventre}, 
  linktoc=page,
  pagebackref,
  colorlinks=true,
  citecolor=PineGreen,
  linkcolor=PineGreen,
  linkbordercolor=PineGreen,
}


% Internal References

\usepackage[inline,shortlabels]{enumitem}

\numberwithin{equation}{section} 
\numberwithin{figure}{section}

\usepackage[nameinlink,capitalise,noabbrev]{cleveref}

\crefname{equation}{}{} % get \cref to behave as \eqref

% \theoremstyle{plain} % bold name, italic text
\newtheorem{theorem}[equation]{Theorem}%
\newtheorem*{theorem*}{Theorem}%
\newtheorem{lemma}[equation]{Lemma}%
\newtheorem{proposition}[equation]{Proposition}%
\newtheorem{corollary}[equation]{Corollary}%
\newtheorem{conjecture}[equation]{Conjecture}%
\newtheorem*{conjecture*}{Conjecture}%
\newtheorem{claim}[equation]{Claim}%


\theoremstyle{definition} % bold name, plain text
\newtheorem{definition}[equation]{Definition}%
\newtheorem*{definition*}{Definition}%
\newtheorem{example}[equation]{Example}%
\newtheorem*{example*}{Example}%
\newtheorem{remark}[equation]{Remark}%
\newtheorem{notation}[equation]{Notation}%
\newtheorem{convention}[equation]{Convention}%
\newtheorem{assumption}[equation]{Assumption}%

\newtheorem{exercise}{Exercise}
\newtheorem{question}[exercise]{Question}

% ---------- macros
\newcommand{\set}[1]{\left\{#1\right\}}%
\newcommand{\sets}[2]{\left\{ #1 \;|\; #2\right\}}%
\newcommand{\longto}{\longrightarrow}%
\newcommand{\into}{\hookrightarrow}%
\newcommand{\onto}{\twoheadrightarrow}%

\usepackage{harpoon}
\newcommand{\vect}[1]{\text{\overrightharp{\ensuremath{#1}}}}

\newcommand{\del}{\partial}%

\newcommand{\ki}{\chi}
\newcommand{\ksi}{\xi}
\newcommand{\Ksi}{\Xi}

% %%%%%%%%%%%%%%%%%%%%%%%%%%%%%%%%%%%%%%%%%%%%%%%%%%%%%%%%%%%%%%%%%%%%%%%%%%%%%%%%%%%%%%%%%%%%%%%%%%%%

\begin{document}

\begin{center}
        \textbf{\Large Math 135, Calculus 1, Fall 2020}\\[10pt]
        {\large Weekly Quiz 10-07}
\end{center}

\thispagestyle{empty}

\renewcommand{\thesection}{\Alph{section}}

Show all work: clearly indicate your answer and the reasoning used to arrive at the answer.
Unsupported answers may not receive full credit.
$ $

\begin{exercise}
        Let $f(x)$ be the function
        \[
                f(x) = 
                \begin{cases}
                        2x-1 \qquad & \mbox{if $x<-1$}\\
                        cx & \mbox{if $-1 \leq x \leq 2$}\\
                        |x-2| & \mbox{if $x > 2$}.
                \end{cases}
        \]
        \begin{enumerate}[(a)]
        \item Find the value of $c$ that makes $f(x)$ left-continuous.
                \vfill
        \item Find the value of $c$ that makes $f(x)$ right-continuous.
                \vfill
        \end{enumerate}
\end{exercise}

\begin{exercise}
        Consider the functions $g(x)$ and $h(x)$ with the following graphs:
        
        \begin{minipage}{0.5\textwidth}
                \begin{center}
                        \includegraphics[width=2.5in]{10-07P_g.png}
                \end{center}
        \end{minipage}
        \begin{minipage}{0.5\textwidth}
                \begin{center}
                        \includegraphics[width=2.5in]{10-07P_h.png}
                \end{center}
        \end{minipage}
        
        Compute $\displaystyle\lim_{x \to 0}(g(x)+h(x))$.
        \vfill
\end{exercise}

\end{document}


