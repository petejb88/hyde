\documentclass[11pt]{amsart}

\usepackage[pagebackref, colorlinks, citecolor=PineGreen, linkcolor=PineGreen]{hyperref}
\hypersetup{
  final,
  pdftitle={20F Math 135 Syllabus, Bonventre},
  pdfauthor={Peter Bonventre},
  linktoc=page
}
\urlstyle{same}

\usepackage{amsmath, amsthm}% {amsfonts, amssymb}


% ------ New Characters --------------------------------------

\usepackage[T1]{fontenc}
\usepackage{MnSymbol}
\usepackage[
cal = cm,
bb = ams,
frak = euler,
scr = rsfs
]{mathalpha}

\usepackage[normalem]{ulem}% underlining
\usepackage{bbm}% more bb

%\usepackage{dsfont}% double strike-through
% \usepackage{upgreek}

\usepackage{csquotes}


%----- Enumerate ---------------------------------------------
\usepackage[inline,shortlabels]{enumitem}% % can use \begin{enumerate*} for inparaenum


% ---------- Page Typesetting ----------
\usepackage[final]{microtype}
\usepackage{relsize}
\usepackage[margin=1in]{geometry}
\usepackage{multicol}


\pagestyle{plain}
\usepackage{tikz}

\setlength{\parindent}{0em}
\setlength{\parskip}{1ex}



\begin{document}

\thispagestyle{empty}

\begin{center}
{\Large
  Math 135: Calculus 1} \\
Fall 2020\\
Zoom, MWF 915-1005 (Section 02) or 215-305 (Section 06)
\end{center}


\normalsize
\begin{multicols}{2}
Instructor: Dr. Peter Bonventre\\
E-mail: pbonvent@holycross.edu\\

Office Hours:  TBD\\
\end{multicols}
\vspace{-2.7em}
Course Website: \url{https://petejb88.github.io/teaching/20F_Math135} and \href{https://hc.instructure.com}{Canvas}


\subsection*{Format} Online. Classes will be conducted on Zoom, with some use of internet tools, such as Google Docs, Jamboards, and Canvas.


\subsection*{Textbook}
\textit{Calculus: Early Transcendentals, 4th Edition} by Jon Rogawski, Colin Adams, and Robert Franzosa.
Chapters 1 -- 5 (up to Section 5.4).\\
You will need access to WebAssign. This comes bundled with the physical book from the College bookstore, but can also be purchased entirely electronically on the WebAssign website (and comes with an digital version of the text).



\subsection*{Is this the right Calculus course for me?}
This course is designed for students whoL
\begin{enumerate*}
\item are interested in majoring in Mathematics, Computer Science, Physics, Biology, Chemistry, Economics, or Accounting;
\item have \textbf{not} recieved a 4 or 5 on the AP Calculus exam for either AB or BC.        
\end{enumerate*}
For more information, see the
\href{https://mathcs.holycross.edu/~little/2019MATH135/Fall2019CalcLetter.pdf}{Holy Cross Math Department website}
for more information.


\subsection*{Course Description}
Math 135 is an introduction to the tools and techniques of Calculus.
In particular, the main focus of the course is the study of real-valued functions of a single variable.
The subject will be approached from both a conceptual and a computational viewpoint.
Rather than just learning a set of formulas, techniques, and algorithms,
the theory and applications of calculuus will be centraol to our study.
Additionally, the course will also require you to effectively communicate your solutions.

You will investigate the following ``big questions'':
\begin{itemize}
\item What are common functions used to model the change in one quantity or value when it is determined by another quanitity or value?
        % You will be able to:
        % \begin{enumerate}[(a)]
        % \item use common functions, such as polynomials and rational functions, trigonometric functions, exponential functions, root functions, and their inverses, to model real-world phenomena.
        % \item apply functional relationships such as composition, inversion, and arithmetical operations to solve problems.
        % \item use various representations of functions, such as symbolic expressions, graphs, and tables, to solve problems.
        % \end{enumerate}
\item What functions can we use to model smoothly-changing motion? For an object in motion, how do we measure the change in position for that object at a given instant in time?
      %   You will be able to:
      %   \begin{enumerate}[(a)]
      % \item solve problems involving average velocities.
      % \item use limits to solve problems involving instantaneous velocity.
      % \item learn the definition of continuous function, understand key properties of continuous functions such as the intermediate value theorem, and apply their knowledge to solve problems related to continuity.
      % \end{enumerate}
\item What are the important mathematical properties of functions that model smoothly-changing motion? What mathematical techniques can we use to analyze those functions and develop models with them?
        % You will be able to:
        % \begin{enumerate}[(a)]
        % \item state the definition of the derivative and explain its relationship to computing instantaneous velocity.
        % \item use the derivatives of common functions to solve problems.
        % \item state properties of derivatives, such as the product and quotient rules and chain rule, and use these properties to solve problems.
        % \item use implicit differentiation and related techniques to solve optimization problems.
        % \item state and apply the mean value theorem.
        % \item  state and apply L'Hopital's theorem.
        % \end{enumerate}
\item What phenomena can we model using derivatives and elementary functions?
        % You will be able to:
        % \begin{enumerate}[(a)]
        % \item solve problems involving exponential growth and decay.
        % \item solve problems involving related rates.
        % \item solve optimization problems.
        % \end{enumerate}
\item For an object that is continuously changing position, how do we determine the total change of position during a period of time?
        How do we compute the area of a two-dimensional figure with a curved boundary?
        % You will be able to:
        % \begin{enumerate}[(a)]
        % \item use Riemann sums to approximate net change and areas of curved figures.
        % \item find antiderivatives for elementary functions.
        % \item state the Fundamental Theorem of Calculus.
        % \item evaluate definite integrals using (i) limits of Riemann sums and (ii) the evaluation of anti-derivatives.
        % \item Evaluate indefinite and definite integrals using substitution.
        % \end{enumerate}
\item How can we use polynomials to approximate more complicated functions?
        % You will be able to:
        % \begin{enumerate}[(a)]
        % \item find the linear approximation to a function at a point and use it to solve real-world problems.
        % \item state the definition of the Taylor polynomial for a function.
        % \item use Taylor polynomials to approximate irrational numbers and approximate definite integrals.
        % \end{enumerate}
\end{itemize}

By the end of the semester, you will be able to:
\begin{itemize}
\item set up and solve word problems
\item explain the results and context of your computations
\item interpret formulas and processes
\item clearly communicate your solution process.
\item collaborate and produce work with others
\item investigate new definitions and theorems with examples and counterexamples
\end{itemize}


\subsection*{The ``Flipped'' Classroom}

To encourage the growth of these skills and learning outcomes,
this course will be using a ``flipped classroom'' style of instruction.
The majority of our class time together will consist of working on guided worksheets or projects in small groups.
There will be less traditional lecturing so that active student learning is the primary focus.

While working in groups:
\begin{itemize}
\item \textit{Share responsibility for making sure all voices are heard:}
        If you tend to have a lot to say, make sure you leave sufficient space to hear from others. If you tend to stay quiet in group discussions, challenge yourself to contribute so others can learn from you.
\item \textit{Understand that we are bound to make mistakes in this space:}
        Everybody (myself included!) does so when approaching complex tasks or learning new skills. In particular, you are invited to step outside your comfort zone!
\end{itemize}


\subsection*{Homework and Assessments}

This flipped classroom approach will be scaffolded and complemented by the out-of-class assignments.

There will be three types of homework:
\begin{enumerate}[(1)]
\item \textbf{Daily Homework.}\\
        In order to prepare for active in-class learning,
        there will be an assignment \textbf{due at the start of every class}.
        These will take the form of \textit{Modules} on the Canvas page for this course.
        Typically, these will include
        an introduction to the topic of the day, usually by either
        a short video lecture or
        a link to an online visualization,
        and a short Canvas quiz on the introduced material.
\item \textbf{WebAssign Weekly Homework, due on Wednesdays at the start of class.}\\
        There will be weekly assignments on \href{https://www.webassign.net}{WebAssign}.

        To access WebAssign, you will need a \textit{class key}:
        \[
                \mbox{Section 002: \textbf{holycross 7770 3656}},
                \qquad
                \mbox{Section 006: \textbf{holycross XXXX XXXX}}.
               
        \]
        You will have free access to WebAssign for two weeks. After that, you will need to purchase access,
        either bundled with a physical textbook, or online directly through WebAssign.

        
\item \textbf{Written Homework, due every Friday.}\\
        The written homework will consist of a small number of more-involved problems.
        These will be submitted as Canvas Assignments.
\end{enumerate}

Additionally:

\begin{enumerate}
\item[(4)] \textbf{Quizzes, due every Monday.}\\
        Each quiz will be open on Canvas between 8am and 6pm EST every Monday, and will be time-limited.
\item[(5)] \textbf{Exams.}\\
        There will be none.
\end{enumerate}

Late assignments will \textbf{not} be accepted.
However, only the 10 best scores of each type (WebAssign, Written, Quiz) will count towards your final grade.



\subsection*{Grading}
Grades will be assigned based on the following scheme:
\begin{multicols}{2}
      Participation  --- 40\% \\
      WebAssign --- 20\% \\
      Quizzes --- 20\%\\
      Written Homework --- 20\%
\end{multicols}


\subsection*{Course expectations}
I know this is a weird and difficult time for us all.
I expect myself to work hard to make this class effective and flexible, and I expect you to do the same.

Math 135 students are expected to complete their assignments, come to class on time and ready to participate and engage with the material and their fellow classmates.

Additionally, you are responsible for announcements made in class, as well as any emails sent to your UK email account or announcements on the course website.

\subsubsection*{Attendance}
Attendance is required.
That being said, I expect there may be times where you are not able to make it to class,
for a potential variety of health or personal reasons.
If you must miss class, due to an illness or other pressing circumstance, please let me know as soon possible.
I will not ask for medical documentation, and naturally the Class Deans are not going to be able to provide excused absences.
Instead, I will trust your judgement and voice in these matters,
and expect that you will take ownership of this trust and act responsibly.

As listed above, participation is \textbf{40 percent} of your grade this semester.
That includes the Daily Homework and attendance, as well as in-class work and engagement.
Your effect and energy into our class time together is essential to this course.
You get out what you put in; this grading scheme codifies that numerically.

\subsubsection*{Cameras}
You may have considerations that will prevent you from keeping your camera on during our synchronous meetings, including internet speed or access issues, family responsibilities, or personal discomfort, so you may absolutely leave your camera off if you want or need to do so. To the extent that you are comfortable and able to turn your camera on, though, please feel free to do so (and be mindful of what's within your camera's view or which virtual background you're using!). This will help us to create a sense of connection and community in our class and encourage engagement with and trust in one another.

Please try, however, to mute your microphone unless you are actively speaking or would like to offer a thought or question. This is to ensure that we give due focus to whoever is speaking and to avoid being distracted by unintended background noise.


\subsection*{Academic Integrity}
The Department of Mathematics and Computer Science has drafted a
\href{https://www.holycross.edu/academics/programs/mathematics-and-computer-science/node/211581/academic-integrity}{Policy on Academic Integrity}.
% It begins:
% \begin{displayquote}
%         All education is a cooperative enterprise between teachers and students. This cooperation works well only when there is trust and mutual respect between everyone involved. One of our main aims as a department is to help students become knowledgeable and sophisticated learners, able to think and work both independently and in concert with their peers. Representing another person's work as your own in any form (plagiarism or "cheating"), and providing or receiving unauthorized assistance on assignments (collusion) are lapses of academic integrity because they subvert the learning process and show a fundamental lack of respect for the educational enterprise.
% \end{displayquote}
Please read this policy in full. By taking this class, you assume responsibility towards following this policy.
If you cheat in this class, you risk failing the course.

\subsubsection*{Collaboration} Mathematics is an inherently collaborative and social activity. On all of the homework assignments, you are encouraged to work together.
However, the solution you submit for credit \textbf{must be your own work}.
In particular, you should prepare your formal solutions to the written assignments independently,
and you should submit your answers for web homework independently.

You are \textbf{not} allowed to work together on quizzes, nor use books, notes, the internet, etc.



\subsection*{Accommodations}
It is my job to provide all students with an accessible and inclusive learning environment.
Some aspects of this course, the assignments, the in-class activities, and the way the course is usually taught may be modified to facilitate your participation and progress.
As soon as you make me aware of your needs, we can work with the
\href{https://www.holycross.edu/health-wellness-and-access/office-accessibility-services}{Office of Accessibility Services}
to determine appropriate accomodations.
Any information you provide is private and confidential, and will be treated as such.


\subsection*{Advice}
\begin{itemize}
\item Work with others!
\item Attend class, participate, and ask questions.
\item Office hours are a great place to ask questions, go over material, and work through problems.
\item Make an appointment with the \href{https://www.holycross.edu/support-and-resources/academic-services-and-learning-resources/steme-workshop}{STEM+E Workshop} for peer tutoring.
\item Learning is not fast, don't try to rush it. Be patient with yourself.
\item Even if you think you are good at multitasking, you work better when you focus on a single task.
\item Just because the first approach at a problem does not work, does not mean that the second or third will not. Sometimes the first thing you (or I) try doesn't work, but this does not necessarily mean that you do not understand the tools required to solve the problem.
\item Start your homework sets early and work together! You should make major progress over the weekend so that you can ask questions in class and/or office hours.
\item Start the webwork early. The first few will be easy, but they will get harder!
\item Get help when needed! Find people you like working with!
\item Just because the first approach at a problem does not work, does not mean that the second or third will not. Sometimes the first thing you (or I) try doesn't work, but this does not necessarily mean that you do not understand the tools required to solve the problem.
\end{itemize}



\subsection*{Important Dates}
\begin{itemize}
\item Last day to add/drop/audit: September 7
\item Labor Day; classes will be held: September 7
\item No Class Day 1: September 30
\item No Class Day 2: November 3
\item Last day to withdraw with a W: November 24
\item Thanksgiving Break (no classes): November 23--27
\item Last day of Classes: Decemeber 9
\item First day of Finals: December 14
\end{itemize}
\end{document}


